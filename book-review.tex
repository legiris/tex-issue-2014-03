\newpage\nuluj
\pagestyle{coady}

\autorb{DAVID COADY}
\nadpisb{What to Believe Now: Applying Epistemology to Contemporary Issues}
\vyd{Wiley-Blackwell, 2012}{ISBN 978-1405199933}
\uvod{coady.jpg}

In \fntit{What to Believe Now}, David Coady applies epistemology to some topics of everyday relevance, to rumours, conspiracy theories, the blogosphere, democracy and experts. A~common theme is an emphasis on the political implications of these different sources of, or contexts for, knowledge. 

An introduction, which could be skipped without harm by more pragmatic readers, surveys some theory, taking Alvin Goldman's ``veritism" as a~starting point. Coady touches on the balance between avoiding error and ``being well informed", proceduralism versus consequentialism, doxastic voluntarism (the idea that we have some responsibility for our beliefs), and the possibilities of virtue epistemology. He argues that applied epistemology is a~branch of applied ethics, and one that is important for social policy. 

Self-contained chapters then address the five main topics.

Looking at the role of experts, Coady uses climate change as an example. There are reasons for skepticism about experts, such as the value of intellectual autonomy and the risk of information cascades, but rejection (or ``reduction") of expert testimony is unsustainable. What do we do when experts disagree? It can make sense to simply go by numbers (some Bayesian probability analysis here is not particularly interesting), with non-independence balanced by the strength of peer evaluation, and we need to beware of rhetoric and take into account evidence of dishonesty or unconscious conflicts of interest. Expertise is both widespread and specific: there is no such thing as ``a scientific expert" per se, for example -- geologists and climate scientists have different domains of expertise -- and in this regard science resembles morality, with which Coady draws some comparisons. 

An epistemic approach to democracy tackles some fairly traditional questions: Are votes statements, and if so what do they say? What epistemic authority do elections have -- does it make sense to change one's own opinion if an election goes the other way? And so forth. Coady emphasizes the importance of knowledgeable voters and sets his conception of epistemic democracy in the context of similar ideas of deliberative democracy. 

Coady defines rumours as unofficial communications that have spread through a~large number of informants, distinguishing them from urban legends, propaganda and other variants, and argues that they do not deserve their bad reputation. There are circumstances when official sources are unreliable or unavailable, and transmission can improve as well as degrade information quality, as people preferentially forward rumours from trusted communicators and modify or update their content using their own knowledge. Where most participants have some relevant knowledge, rumours can be highly accurate. In the Second World War, the US military ran counter-rumour operations, concerned not that rumours about troop movements were wrong, but that they were too accurate and could easily spread to the enemy. 

The widespread modern objection to conspiracy theories qua conspiracy theories originated with Karl Popper. Coady works through the obvious -- once considered -- problems with this: conspiracies are not uncommon, can succeed, and have important consequences. There are indeed many stupid conspiracy theories, but their problem is that they are stupid, not that they are conspiracy theories. Coady concludes with the suggestion that proper consideration of conspiracy theories is, ironically, necessary for anything like Popper's Open Society to function. 

Finally, Coady turns to the blogosphere and how it compares with the traditional media (comparing it to the legal process or scientific research is misguided). He looks at journalism as a~profession, the notion of ``balance", the different kinds of filtering employed, claims of ``parasitism", and so forth. There are advantages in low barriers to entry and interactivity and, allowing benefits to ``promoting true belief" and not just to avoiding error, the epistemic consequences of the Internet are really not so bleak. 

A conclusion touches briefly on Wikipedia, torture and political skepticism, while a~postscript presents a~non-privacy argument against extensive use of CCTV camera monitoring. 

In so far as a~political stance can be distinguished in \fntit{What to Believe Now}, it is broadly anarchist, in that Coady argues for the merits of decentralised and distributed sources of knowledge. He is also skeptical about the epistemic reliability of the state and institutions, certainly much more so than theorists such as Cass Sunstein. This comes out most clearly, perhaps, in the chapters on rumours and conspiracy theories and in the postscript. 

\fntit{What To Believe Now} is aimed at more practical problems than most epistemology, but it is still rather abstract: Coady is an academic writing as a~participant in ongoing academic debates. In particular, he often appears to be having a~kind of conversation with Alvin Goldman, whose ideas have a~high profile in almost all the chapters. It is certainly possible to imagine a~considerably ``more applied" applied epistemology, which might consider practical guidelines for evaluating rumours or distinguishing more reliable Wikipedia pages from less reliable ones, among other matters. 

With the possible exception of some of the introductory material, however, everything in \fntit{What To Believe Now} is accessible without a~background in epistemology. Since it addresses topics of considerable importance, it should command, if not a~mass audience, then one that reaches well outside the narrow confines of academic philosophy. Those particularly likely to find it useful include political theorists, students of social networks, and perhaps some policy makers. 
