\newpage\nuluj
\pagestyle{usiak}

\logo

\nadpis{The International Court of Justice and the Legality\\of UN Security Council Resolutions}

\autor{Jaroslav U�iak\pozn{Faculty of Political Science and International Relations, Matej Bel University in Bansk� Bystrica, Kuzm�nyho\;1, 974 01 Bansk� Bystrica, Slovak Republic. E-mail: Jaroslav.Usiak@umb.sk.}, �ubica Saktorov�\pozn{Faculty of Political Science and International Relations, Matej Bel University in Bansk� Bystrica, Kuzm�nyho\;1, 974 01 Bansk� Bystrica, Slovak Republic. E-mail: Lubica.Saktorova@umb.sk.}}
\linka{3ex}

\cast{Abstract}
Through the United Nation's Charter, the UN Security Council represents the most powerful executive institutional body in the field of collective security. Moreover, its ultra vires acts may have distinct legal consequences. Accordingly, questions arising from these facts are whether such a~large scope of competences could be abused, what are the limits of the executed power and above all, affirmation of the legality of the actions of Council. Predominantly by means of the analytical method as well as a~case study of the Lockerbie case, the present study provides the related argumentative discourse. Notwithstanding the fact that the decisions of the Council appear to be without any limitations, it is obliged to act within the purposes and principles of the UN Charter. The opinion of the International Court of Justice on the legality of the SC's actions in the field of collective security are still, however, indirectly expressed through its general function.\\

\cast{Keywords}
United Nation's Security Council, International Court of Justice, UN Security Council Resolutions, Collective Security, UN Charter
\linka{3ex}

%\vspace*{-4mm}
\podnadpis{Introduction}
The relationship between the Security Council (``SC'' or ``Council'') and the International Court of Justice (``ICJ'' or ``Court'') is established by the Charter of the United Nations as well as by the Court's Statute. The International Court of Justice is seen as the principal judicial organ of the United Nations\pozn{Article 92, UN Charter.} and is a~body to which States may refer their legal disputes, under Article 36 of the UN Charter; the Security Council can make a~recommendation for action to be taken. Another role of the International Court of Justice is to provide the General Assembly and the Security Council with an advisory opinion on any legal question\pozn{Article 96, UN Charter.}. Currently, there are no express provisions in the Charter, nor in the Statute of the Court determining its power of review over political actions, even though this concept among the principal governing organs is often seen in different systems (Nigel, 2005, p.\;216).

This study will examine whether the International Court of Justice of the United Nations is the proper body for reviewing the legality of the Security Council's actions; first, through the constitutional framework by the analogy of judicial review of the political decision with legal consequences and its restrictions in different legal systems. The second part will assess the function of the ICJ with regard to the Security Council's primary responsibility in maintaining international peace and security, supported by the case study of the \fntit{Lockerbie} case\pozn{Questions of interpretation and Application of the 1971 Montreal Convention Arising from the Aerial Incident at 1992 (\fntit{Libyan Arab Jamahiriya v. United States of America}) ICJ Reports (1992) PM 1992/9.\\Questions of interpretation and Application of the 1971 Montreal Convention Arising from the Aerial Incident at 1992 (\fntit{Libyan Arab Jamahiriya v. United States of America}) ICJ Reports (1992) PM 1992/8.} that, since the establishment of the United Nations, brought the given question back to wide discussion. The third part will recognise the realistic implications of the ICJ being in a~position to review the legality of the SC's actions in the field of collective security.

\podnadpis{The basis for the constitutional parallels}
In the common law system judicial review has officially obtained the explicit legal frame in the famous \fntit{Marburv v. Madison} case\pozn{\fntit{Marbury v~Madison} (1803) 5 US (1 Cranch) 137.}. The principle of review of the constitutionality of acts of the U.S. Congress was not expressed in the Constitution; similarly, judicial review of the Council's acts has no legal support in the UN Charter. Where there are two colliding norms that are impossible to apply consistently alongside each other, the authority is obliged to choose the one with the higher status. Thus the judicial organ is bound to use legislation rather than an executive decree and the constitution rather than legislation (Akande, 1997, p.\;326). The constitutional framework assumes the separation of power and its consequent system of control among the principal governmental organs. This concept has evolved over centuries from, primarily, philosophical ideas based on the doctrine that every democratic governing system should build its rules on equally assessed branches of executive, legislative and judicial power. In the 17th century, John Locke and later Charles Baron de Montesquieu developed the theory of the horizontal position of governmental organs ensuring mutual balance and control (Loveland, 2012, p.\;54). Their works had a~profound influence on theoretical analyses and became the background for the different national legal systems in which the judicial bodies, in specific domestic traditions, particularly the constitutional court, was provided with the capacity to review the legality and compatibility of the decision of the executive organs \fntit{vis-\`a-vis} the rule of law (Akande, 1997, p.\;309).

%\clearpage\newpage

This constitutional point of view often provides a~clear analogy to simply use the same concept with regard to international judicial bodies and consequently would presume their capacity of reviewing the legality of actions or decisions of the SC by the ICJ. However, the practice of judicial review in domestic legal systems includes exceptions, particularly those regarding the actions of heads of state representatives related to state security or defence, e.g. proclamations of a~state of war (Drgonec, 2012). Considering this paradigm, judicial systems on the international level appear to follow national patterns in relation to the executive. Even though the national judiciary usually include the function of judicial review, in order to ensure prompt and effective action related to maintenance of peace and security within the state, these enactments are not subject to judicial review.

However, consideration of the vertical strength or hierarchy of acts and their subsequent application is indeed limited. In the United Kingdom, control of the armed forces cannot be challenged in court.\pozn{\fntit{Chandler v~DPP} (1964) AC 763.} Decisions that involve national security in the sense of declaring war or deploying armed forces are exclusively within the authority of the executive power and it is accepted that decisions to take military action are beyond the court's purview.\pozn{\fntit{R v~prime Minister of the UK} (2002) EWHC 2759 QB.} A~subsequent decision of the Crown court confirmed the principle: the courts will not review the legality of foreign policy in terms of declarations of war or the deployment of armed forces and have identified the matter as not justiciable.\pozn{\fntit{R v~(1) Philip Pritchard and Others} (2004) EWCA Crim, 1981.} Similarly, in other countries, the constitutional court is in its function often seen, with explicit exceptions, as being excluded from reviewing executive acts. These are usually directly implemented in the constitutional acts of a~state. The constitutional court consequently finds the scope of its review, with regard to legislative/executive acts, limited. The restrictions are generally related to decisions of executive organs that in cases of state emergency deal with national security (Hor, 2005, p.\;273).

In national legal cultures and most democratised countries, the courts in general, or the constitutional courts, possess the capacity to also function as a~legal surveillance tool for the legality, validity and compatibility of acts passed by the government. Explorations that compare the United Nations and its rules of function to various domestic systems might appear to have logical, probably ideological grounds. This at first seems to make any analogy of the International Court of Justice to national courts impossible; the United Nations should not be compared to any national bodies or legal societies. The UN is a~political organization and the United Nation Charter is the act constituting this organisation according to the conditions that member states have agreed to and complied with (Schweigman, 2001, p.\;15). The Security Council and the International Court of Justice are the organs of the United Nations and the scope of its function are explicitly constituted. Their unique purposes differ from any other, though the applications of any kind of framework, concept or structure in comparative sense appear to be irrelevant.

The second parallel, a~concrete comparison of the system of judicial review of the legality of the Security Council's action to domestic reviewing procedures, finds obvious gaps. Actions in the scope of collective security under Chapter VII of the UN Charter would analogically fall into matters that are in most democratic countries excluded from the competence of the organs providing judicial review. The state's decisions and subsequent actions taken related to national security, the use of the armed forces, and military action are often protected by the prerogative clause and are particularly considered to be non-justiciable (Matheson, 2004, p.\;615).

Through this analysis, it can be determined whether the ICJ is the proper body to review the legality of the actions of the Security Council, although the temptation to apply philosophical ideas and conventional proposals, or any potential constitutional parallel, would not be just. Review of the legality of actions in the field of collective security would be most probably, analogically to national practice, excluded from the extent of the powers of the International Court of Justice, while even in the most democratised cultures, the consequences of executive action are preferably judged politically rather than legally by courts (Loveland, 2012, p.\;114).

\podnadpis{The role of the International Court of Justice in relation to the Security Council's function}
The role of the International Court of Justice attained its legal ground in 1945 by means of the Charter of United Nations, whereby Chapter XIV assesses its function as the principal judicial organ of the United Nations. The ICJ acts in accordance with the UN Charter and an annexed Statute of the International Court of Justice in two different fields: non-binding advisory opinions and continuous cases with mandatory consequences for the states involved. Procedures for determining the legality of the actions of United Nations organs by the International Court of Justice had not yet obtained a~practical and explicit legal framework.\pozn{\fntit{Certain Expenses of the United Nations etc}. ICJ Rep (1962) 62/20 AO 168.}\\

\fntit{Advisory opinions}

Firstly, the ICJ is entitled to provide the General Assembly and the Security Council with an advisory opinion on any legal question. Moreover, a~request for an advisory opinion might also be made by the other organ of the United Nations and specialised agencies with authorisation of the General Assembly on legal questions arising from the scope of their activities.\pozn{Article 96, UN Charter.} In the case of the General Assembly and the Security Council, such restrictions do not exist. The broad meaning of `any legal question' could, however, suggest the mutual questioning of the legality of the acts of both principal bodies. The legal ground for the possibility to review the legality of actions of the Security Council consequently seems apparent, according to this article (Akande, 1997, p.\;328). Realistically, the availability of this option would necessarily require the full force of political will.

%%%% v teto casti nejak nesedi uvozovky ...In the legal...
However, a~request to enquire as to the legality of political decisions within the United Nation's corpus was made in the \fntit{Expenses case}, where the Court noted: In the legal system of States, there is often some procedure for determining the validity of even a~legislative or governmental act, but no analogous procedure is to be found in the structure of the United Nations.\pozn{Article 96, UN Charter.} A~very similar conclusion with a~more precise point was stressed in the \fntit{Namibia case}\pozn{The ICJ delivers its Advisory Opinion on the Legal Consequences for States of the Continued. \fntit{Presence of South Africa in Namibia (South West Africa) notwithstanding Security Council Resolution 276} (1970).\\ICJ Rep (1971) 71/10, AO 19.}, whereby the Court stated that it `does not possess powers of judicial review or appeal in respect of the decisions taken by the United Nations organs concerned and also that the subject matter of the request for advisory opinion cannot be formed by the question of validity or conformity with the Charter of General Assembly Resolution 2145 and related Security Council decisions'.

Even though the ICJ's opinion on the capacity to review the legality of actions was not positively expressed, the Court thereafter examined the Security Council's power to issue the particular resolution and the entitlement of the General Assembly to terminate the mandate by means of the abovementioned resolution 2145.\pozn{ICJ Rep (1971) 71/10, AO 51.} The way for the ICJ to determine whether the political decision is valid -- notwithstanding its lack of an exact right to review -- was thus potentially demarked.\\

\fntit{Contentious cases}

The jurisdiction of the International Court of Justice subsumes cases which the parties refer to it and matters especially provided for in the Charter of the United Nations or in other treaties and conventions in force.\pozn{Article 36, ICJ Statute.} The Security Council can thus make a~recommendation that a~legal dispute should be, as a~general law, referred by the parties under the Court's jurisdiction.\pozn{Article 36(3), UN Charter.} The decisions of the ICJ are, in contentious cases, binding solely on an inter-party basis.\pozn{Article 94, UN Charter.}

When examining the aptitude of the ICJ as the proper body to review the legality of the SC's actions, it is worth noting the case that brought this issue into wider discussion.\\

\fntit{The Lockerbie case}\pozn{Questions of interpretation and Application of the 1971 Montreal Convention Arising from the Aerial Incident at 1992 (\fntit{Libyan Arab Jamahiriya v. United States of America}) ICJ Reports (1992) PM 1992/9.}

In December 1988, a~Pan American aircraft exploded over the village of Lockerbie, Scotland, and caused the death of 259 people on board and 11 on the ground. While the plane and most passengers were American, the people killed on the ground were British and the incident happened in the territory of the United Kingdom; the parties of the dispute and the jurisdiction over the case were presumed. In November 1991, US and UK bodies issued an indictment accusing two Libyans nationals, along with a~warrant for their arrest and request for their extradition. Libya refused to comply with reference to the 1971 Montreal Convention as the general rule of jurisdiction in cases of claims to extradite nationals, asking for arbitration according to Article 14/1 of the Montreal Convention. In January 1992, the Security Council adopted resolution 731, ordering Libya `to provide a~full and effective response' to the requests of the UK and US. Two months later, Libya applied to the ICJ asking for a~judgement that would declare a~breach of the obligations of the US and UK under the Montreal convention and also for provisional measures to prevent the extradition. Before the delivery of the ICJ's judgement, the Security Council adopted resolution 748, under Chapter VII of the UN Charter imposing the mandatory sanctions in the event that Libya failed to comply. Requests for provisional measures were dismissed according to the binding and predominant effect of the Security Council's decisions (Martenczuk, 1999, p.\;517--521).

The UN Security Council is, in accordance with the UN Charter, obliged to act in accordance with the principles and purposes of the United Nations and should, according to Article 36.2, take into consideration any procedures which have already been adopted by the parties for the settlement of a~dispute. In the meaning of the legal disputes of the \fntit{Lockerbie case}, all concerned parties agreed to Article 14.1 of the 1971 Montreal Convention, according to which the proper action for the Security Council would have been to call for arbitration. But yet again, it is the UN Security Council itself that considers and decides the methods and necessary measures to be undertaken when threats to international peace and security occur.

Regarding the jurisprudence of the UN Security Council in relation to the fight against terrorism, resolution 731 seems to be outside regular customs. In terms of the allegation of the individuals to have any involvement in the terrorist act, the previous resolutions dealing with the terrorism were strictly constrained to support the convention regime or the bilateral treaties. According to the content of the resolution; contrary to the previous practice, whereby the legal background for adopting the act is essential, the explicit legal basis for the resolution 731 is missing. The resolution does not, however, indicate any breach or threat of international peace and security, nor an act of aggression. Furthermore, Libya was faced with complying with actions introduced by means of references, not directly, as seen in past Security Council practice (Krishnamurthy, 2014). Insight into the factual and legal content makes resolution 731 disputable and provides room for different argumentation. 

Examination of the measures provided by the UN Security Council in cases where a~certain event has been considered to jeopardize world peace and security appears to provide no legal pattern, nor consistency (Krishnamurthy, 2014). In different cases the UN Security Council calls for condemnation, arbitration, the use of economic sanctions or other tools, until it comes to international intervention and the use of direct force (Malone, 2004, p.\;211). In the \fntit{Lockerbie case}, for instance, the Security Council, with resolution 748, called for the prohibition of air flights, prohibition of the sale or supply to Libya or its nationals of military weapons, ammunition, vehicles, equipment or parts; and above all, the prohibition of the supply to Libya or its nationals of military training or advice.\pozn{S/RES/748 (1992).} The UN Charter requests the determination of a~threat to the peace, a~breach of peace or an act of aggression by the Security Council to apply the measures under article 41 and~42. Literally, neither resolution 731 nor 748 explicitly reference satisfying this rule. The application of such international sanctions by the Security Council under the Chapter VII related to this case remains unclear (Aoude, 2014).

The opposite attitude to the situation could be seen in the \fntit{Corfu Channel case}\pozn{Corfu Chanel case (\fntit{Albania v. United Kingdom}) (1948) ICJ Rep 1948/1.}, whereby the Security Council, according to Article 36.3, in making recommendations to Albania and the United Kingdom under Chapter VI, referred the case to the International Court of Justice. Important determinations were made by the ICJ in the \fntit{Nicaragua case}\pozn{Military and Paramilitary Activities in and against Nicaragua (\fntit{Nicaragua v. United States of America}), ICJ Rep (1986) 86/6.}. The Court stated that a~breach of the customary principles of international law, in the sense of infringement of the territorial integrity of Nicaragua, had taken place and considered the actions of the United States as military and paramilitary help to the rebel group, ergo the illegal use of force against the Government of Nicaragua; nevertheless, it disregarded US assistance as collective self-defence. According to Article 2.4 of the Charter of United Nations, Member States shall refrain in their international relations from the threat or use of force against the territorial integrity or political independence of any state. Firstly, the ICJ provided a~legal response, which might contribute to illuminating its role in similar  future situations (Hossain, 2010, p.\;91). Secondly, through the judgement of the \fntit{Nicaragua case}, another issue had been raised: the demarcation of functions between the General Assembly and the Security Council according to Article 12 of the UN Charter cannot be seen as a~paradigm, while no such provision in the Charter defines the function between the ICJ and the Security Council. `The Council has functions of a~political nature assigned to it, whereas the Court exercises purely judicial functions. Both organs can perform their separate but complementary functions with respect to the same events.'\pozn{S/RES/748 (1992).}

The International Court of Justice in these cases acted in accordance with the sphere of its authority. The basis providing the Security Council's decisions and actions with the full support of the law are constituted within the United Nations Charter, though even the highly conspicuous actions in a~name of collective security might be given ``political legitimacy''. The Security Council has exuberant powers to determine the existence of threats or a~breach of international peace and security, and such determination currently appears to remain beyond doubt. Nevertheless, the International Court of Justice has clarified its competence to act in terms of threat or a~breach of international peace and security.

\podnadpis{Implications of the reviewing function of the International Court of Justice}
Examination of whether the International Court of Justice is or is not a~proper body for reviewing the legality of the Security Council's action in the field of collective security should be at the last stage restricted to the implications for the system of the United Nations as a~whole. The point of departure for this analysis must be based on Article 24 of the UN Charter, under which the member states agree to confer the primary responsibility for maintaining international peace and security on the Security Council. It is acting on their behalf, whereby the limits of such actions are to be seen in accordance with the purposes and principles of the United Nations.\pozn{Article 24 UN Charter.} The concept of collective security in the United Nations is based on the idea of the institutionalised cooperation of states in order to strengthen their mutual protection. The established governmental and representative body, with the exception of the executive power, also possess implied legislative power. Article 25 provides explicit legal grounds for the binding force of the Security Council's decisions, which consequently become the source of international law (Amerasinghe, 2005, p.\;293). In relation to Article 103, which provides the obligations of Members of UN under the UN Charter with the predominant position among the other obligations and international agreements, the actions of the Security Council, in the name of international peace and security with respect to international law, may enjoy the full package of a~legal basis. Moreover, the Security Council shall determine and decide under Chapter VII whether a~situation constitutes a~threat or breach of the peace or act of aggression, and if so, what measures are to be taken to resolve such a~situation.\pozn{Articles 39 41 42 UN Charter.} 

Even though the Charter of the United Nations grants the Security Council a~large scope of authority, it is nevertheless the organ of the international organisation that establishes its predominance, the political function and outstanding character from the agreement concluded by member states with an exactly specified aim (Constantinides, 2005). The Council is not a~sovereign entity with powers above the law. The affirmation of such a~statement came recently from the ICTY Appeal Chamber in the \fntit{Tadi� case}.\pozn{\fntit{Prosecutor v. Tadic} (1996) IT-94-1-AR72 para 28.}  The control mechanisms and limits of the Security Council's power under Chapter VII are often the subject of numerous discussions. The cardinal limits are principles and purposes of the United Nations, particularly those described by the Preamble of the UN Charter and the following Chapter I. These can be further developed into the concept of respecting and abiding the general principles of international law, ius cogens and human rights obligations (Akande, 1997, p.\;317). The other possible method of restriction is the veto power that, despite criticism, represents a~tool to control the Security Council's actions (Bailey, 2005, p.\;201). The main point here is that it is not the goal of the members of Security Council to circumvent the law, nor is there to be any hidden promotion of illegal intent. They must act in accordance with rules which have been internationally settled.

In the \fntit{Lockerbie case}, Libya challenged the legality of the Security Council's decisions on the whole, where it is clear that the UN Charter gives authority only to the Security Council to make such decisions.\pozn{Article 36(3), UN Charter.} Nevertheless, the authority of the ICJ in this case recognises the predomination of the obligations under the Charter over any other amendment of the states, so that the ICJ could only consequently confirm the mentioned status quo.\pozn{Article 36 ICJ Statute, 36, 37.} The decisions under Chapter VII are essentially factual, political and selective in character; there are neither legal reasoning nor the legal standards to reach such a~decision (Amerasinghe, 2005). With reference to the analysis presented in the first part of this study, they are indeed difficult to become subject to the standards of judicial review. In national legal systems, such decisions would typically be regarded as non-justiciable (Roberts, 1995).

\clearpage\newpage

Also, by exercising the functions of the Security Council related to Chapter VII in crisis threshold situations, there is an undisputable need for fast and effective reaction, forasmuch as the decisions must hold enough authority to be accepted by all parties as conclusive and binding. Otherwise, the process of deliberating and reviewing might be likely to block responsiveness to actual breaches of international peace and security and the effectiveness of the Council in such crisis situations. The option to review the legality of actions that flow to guarantees of collective security would ultimately harm the credibility of the Security Council, while any kind of leeway allowing for reversibility would seriously depreciate its function. Furthermore, according to the Security Council's determination, Article 48 of the Charter binds member states to implement the decisions of the Security Council for the maintenance of international peace and security into their legal orders. The stark subsistence of the possibility to review the legality of Council's action by the ICJ would have a~dramatic impact on the whole system of the United Nations, because neither the Charter nor the ICJ Statute bestows on the ICJ any powers to rule out or cancel binding decisions adopted under Chapter VII (Krishnamurthy, 2014). Nevertheless, the explicitly expressed right of the ICJ to review the legality of the Security Council's actions, pursuant to Chapter VII of the UN Charter could lead to another negative effect for the UN system. The adverse judgement of the International Court of Justice discrediting the action or decision of the Council in accomplishment of its primary responsibility might well in turn devalue the UN Charter. The ideologies of constitutionalization or judicialization of the UN system and the factual application of these theories may, from a~long-term perspective, increase the will of the members of the UN to obtain greater influence in the Security Council's functions and procedures. Moreover, these potential needs could possibly grow into serious demands and consequently affect other issues, such as membership, voting process, etc. (Alvarez, 1995, p.\;85).

\podnadpis{Conclusion}
The function of the International Court of Justice is set out in the UN Charter and the Statute of the Court. It is often the subject of argumentation as to whether its actual function is satisfactory and adequate with regard to its potential power as the principal judicial organ of the United Nations. The possibility of extending the scope of its authority will probably evolve according to the particular situations and needs of contemporary society. Any constitutional parallels to emphasise the missing role of judicial review of the ICJ as a~principal judicial organ appear to be irrelevant, while even if there were a~general right to review Security Council actions, it is highly unlikely that the ICJ would really contemplate the judicial review of actions that were already pursued under Chapter VII (Hossain, 2010, p.\;95).

The advisory opinion and continuous cases give the ICJ room to provide for control over the Security Council's actions, even though reviewing capacity is not explicitly expressed in the substantive constituting documents. The merit of consideration and any following measures undertaken to react to a~threat or breach of international peace and security, or an act of aggression under Chapter VII of the Charter, are left to the Security Council. These decisions are purely political and, as far as they appear to be without any limitations, the Security Council is obliged to act within the purposes and principles of the UN Charter, including ius cogens, fundamental principles of international law and indeed human rights obligations. The ICJ is neither a~substitute nor a~body of political legitimisation, though its opinion on the legality of the Security Council's actions in the field of collective security may be indirectly expressed through its function, as it is in the present (Hossain, 2010, p.\;122). However any amendment of the Charter of the United Nations or radical change in the UN system related to the International Court of Justice in the function of Chapter VII would currently be most probably impossible, as to provide the ICJ with a~reviewing function could also, inter alia, trigger a~number of undesirable consequences.\\[6mm]

\cast{References}
Akande, D. (1997). The International Court of Justice and the Security Council: Is There Room for Judicial Control of Decisions of the Political Organs of the United Nations? \fntit{International and Comparative Law Quarterly}, 46(2), 309--343.

Alvarez, J. �. (1995). Theoretical Perspectives on Judicial Review by the World Court. In Hargrove, J. L. (ed.). \fntit{American Society of International Law (ASIL). Proceedings of the 89th Annual Meeting: Structures of World Order}. New York, Washington, D.C.: The Society.

Amerasinghe, G. F. (2005). \fntit{Principles of the Institutional Law of International Organisations}. Cambridge: Cambridge University Press.

Aoude, S. (2014). \fntit{The Lockerbie Legality: What Went Wrong in the International Criminal Law?} Retrieved March 27, 2014, from http://www.mathaba.net/info/locker-leg.htm.

Bailey, S. D., Daws, S. (2003). \fntit{The Procedure of the UN Security Council}. Oxford: Oxford Scholarship Online.

Bowett, D. W. (1997). Judicial and political function of the Security Council and the International Court of Justice. In Saab-Abi G. et al., Fox, H. (ed.). \fntit{The Changing Constitution of United Nations}. London: The British Institute of International and Comparative Law.

Constantinides, A. (2005). An overview of Legal Restraints on Security Council Chapter VII Action with Focus on Post-Conflict Iraq. \fntit{Founding Conference}. Florence: European Society of International Law. Retrieved May 27, 2014, from http://www.esil-sedi.eu/si\-tes/default/files/Constantinides\_0.PDF.

Daws, S., Weiss, T. G. (ed.). (2007). \fntit{The Oxford Handbook on the United Nations}. Oxford: Oxford University Press.

Drgonec, J. (2012). Ochrana �stavnosti v~extr�mnych a~nad�tandardn�ch situ�ci�ch. In \fntit{Aktu�lne ot�zky trestn�ho z�konodarstva}. Bratislava: paneur�pska vysok� �kola, 13--44. 

Fassbender, B. Review Essay: Quis judicabit? The Security Council, Its Powers and Its Legal Control. \fntit{European Journal of International Law}, 11(1), 219--232.

Franck, T. M. (1992). The Power of Appreciation: Who Is the Ultimate Guardian of UN Legalities? \fntit{American Journal of International Law}, 86(2), 519--523.

%\clearpage\newpage

Franck, T.\;M. (1995). \fntit{Fairness in International Law and Institutions}, Oxford: Clarendon Press.

\clearpage\newpage

Geoffrey, R.\;W. (1993). Constitutionalism, Judicial Review, and the World Court. \fntit{Harvard International Law Journal}, 34(1), 1--45. 

Graefrath, B. (1993). Leave to the Court What Belongs to the Court -- the Libyan Case. \fntit{European Journal of International Law}, 4(1), 184--205.

Hor, M. (2005). Law and Terror: Singapore Stories and Malaysian Dilemmas. In Ramraj, V.\;V. et al., (eds.). \fntit{Global Anti-Terrorism Law and Policy}. Cambridge: Cambridge University Press.

Hossain, K. (2010). Legality of the Security Council action: Does the International Court of Justice move to take up the challenges of judicial review? \fntit{USAK Yearbook of International Politics \& Law}, 2010(3), 91--122. 

Klabbers, J. (2009). \fntit{An Introduction to International Institutional Law}. Cambridge: Cambridge University Press.

Krishnamurthy, A. (2014). \fntit{The Lockerbie incident Case: the judicial review by ICJ}. Retrieved May 15, 2014, from http://www.legalserviceindia.com/articles/lock.htm.

Kubinec, M., Poto�n�, L. (2008). Odplatnos� zmluvy o~prevode obchodn�ho podielu. In \fntit{S��asnos� a~perspekt�vy pr�vnej regul�cie obchodn�ch zml�v}. Ko�ice: Univerzita Pavla Jozefa �af�rika v~Ko�iciach, 90--95.

Loveland, I. (2012). \fntit{Constitutional Law, administrative Law and Human Rights: a~critical introduction}.  Oxford: Oxford University Press.

Malone, D.\;M. (2004). \fntit{The UN Security council: from the Cold war to the 21st century}. Boulder: Lynne Rienner Publishers.

Martenczuk, B. (1999). The Security Council, the International Court and Judicial Review: What Lessons from Lockerbie? \fntit{European Journal of International Law}, 10(3), 517--547.

Matheson, M.\;J. (2004). ICJ review of Security Council decisions. \fntit{George Washington International Law Review}, 36(2/3), 615--622.

�berg, M.\;D. (2005). The Legal Effects of Resolutions of the UN Security Council and General Assembly in the Jurisprudence of the ICJ. \fntit{European Journal of International Law}, 16(5), 879--906. 

Roberts, K. (1995). Second-Guessing the Security Council: The International Court of Justice and Its Powers of Judicial Review. \fntit{Pace International Law Review}, 7(2), 281--327.

Sarooshi, D. (1999). \fntit{The United Nations and the Development of Collective Security}. Oxford: Oxford University Press.

Schwarzenberger, G. (1957). \fntit{International Law as applied by International Courts and Tribunal}. London: Stevens \& Sons limited.

Schweigman, D. (2001). \fntit{The authority of the Security Council under Chapter VII of the UN Charter: legal limits and the role of the International Court of Justice}. The Hague: Martinus Nijhoff Publishers.

Talmon, S. (2005). The Security Council as World Legislature. \fntit{American Journal of International Law}, 99(1), 175--193.

White, N.\;D. (2005). \fntit{The Law of International Organisations}. Manchester: Manchester University press. \\

\clearpage\newpage

\cast{Table of Legislation:}
S/RES/731 (1992)

S/RES/748 (1992)

The Statute of the International Court of Justice

The United Nations Charter\\[4mm]

\cast{Table of Cases: }
\fntit{Chandler v~DPP} (1964) AC 763

Corfu Chanel case (\fntit{Albania v. United Kingdom}) (1948) ICJ Rep 1948/1

ICJ Rep (1971) AO 71/10

\fntit{Marbury v~Madison} (1803) 5 US (1 Cranch) 137 

Military and Paramilitary Activities in and against Nicaragua (\fntit{Nicaragua v. United States of America}), ICJ Rep (1986) 86/6  

\fntit{Presence of South Africa in Namibia (South West Africa) notwithstanding Security Council Resolution 276} (1970)

\fntit{Prosecutor v. Tadic} (1996) IT-94-1-AR72 35 ILM 

Questions of interpretation and Application of the 1971 Montreal Convention Arising from the Aerial Incident at 1992 (\fntit{Libyan Arab Jamahiriya v. United States of America}) ICJ Reports (1992) PM 1992/9       

Questions of interpretation and Application of the 1971 Montreal Convention Arising from the Aerial Incident at 1992 (\fntit{Libyan Arab Jamahiriya v. United Kingdom}) ICJ Reports (1992) PM 1992/8

\fntit{R v~(1) Philip Pritchard and Others} (2004) EWCA Crim 1981

\fntit{R v~prime Minister of the UK} (2002) EWHC 2759 QB

The ICJ delivers its Advisory Opinion on the Legal Consequences for States of the Continued.
