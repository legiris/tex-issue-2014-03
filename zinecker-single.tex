\newpage\nuluj
\pagestyle{zinecker}

\logo

\nadpis{German Language Proficiency among Students\\of Business and Management in the Czech Republic\\and its Perception: the Importance of German Language Skills on the Labour Market and the Role of Universities\\in Foreign Language Training}

\autor{Marek Zinecker\pozn{Brno University of Technology, Faculty of Business and Management, Kolejn� 2906/4, CZ-61200 Brno. E-mail: zinecker@fbm.vutbr.cz.} \& Zde�ka Kone�n�\pozn{Brno University of Technology, Faculty of Business and Management, Kolejn� 2906/4, CZ-61200 Brno. E-mail: konecna@fbm.vutbr.cz.}}
\linka{2ex}

\cast{Abstract}
This paper presents the results of an empirical study designed to map German language proficiency among students at Czech universities of business and management. The results of this empirical survey can be summarised as follows. First, the ability of students at Czech universities of business and management to communicate in German is poor, and exceeds the general German language proficiency of the Czech population only to an insignificant extent. Second, the school environment (the opportunity to learn the language, compulsory subject, language study motivation) has a~decisive influence on the respondents' ability to communicate in German. Third, nearly three-quarters of the respondents perceive German as a~language that is very or rather important for their profession and career growth. Fourth, almost two-thirds of the respondents consider the role played by a~university of business and management in the improvement of German language proficiency rather or very important. In conclusion, the study proposes directions for the potential development of the national educational system in the area of German language proficiency of university graduates in business and management in the Czech Republic, with an emphasis on the concept of content and language integrated learning (CLIL). We believe that the survey results are also very important from the point of view of enterprises operating in the Czech Republic because of the very close economic relations between the Czech Republic and German-speaking countries.\\[-1mm]

\cast{Keywords}
Content and Language Integrated Learning (CLIL), German Language Proficiency, Students of Business and Management, Foreign Language Training at Universities
\linka{2ex}

\clearpage\newpage

\podnadpis{Introduction}
For over a~thousand years, the Lands of the Bohemian Crown and the German-speaking environment have been linked ``not only by deep cultural and historical roots but also by economic collaboration'' (isea 2010, M�ller 2003, Riemeck 1997).\pozn{The Lands of the Bohemian Crown were one of the parts of the Holy Roman Empire from the High Middle Ages until its collapse in 1806. Due to German colonization, in particular in border regions, the territory became bilingual by the 13th century or earlier. Between 1806 and the foundation of the Czechoslovak Republic (1918), the Kingdom of Bohemia was one of the Crown lands of the Austrian, or rather Austro-Hungarian, Empire (from 1867). The tragic events of WWII ended the coexistence, peaceful until the 1930s, of the two ethnic groups in the territory of present day Czech Republic (M�ller 2003, Riemeck 1997).} The political and economic transformation of the country after 1989 presumably strengthened the importance of the German language, and German language proficiency became a~prerequisite for success on the labour market in the Czech Republic.\pozn{Since the inception of the Czech Republic (1993), Germany as a~business partner is unequivocally number one in both import and export, as well as the largest direct foreign investor in the Czech Republic. The share of the Federal Republic of Germany in total foreign trade turnover of the Czech Republic oscillates around one-third in the long run (Czech Trade 2012).} On the other hand, due to the globalization of the world economy, English is gaining in importance and is becoming the \fntit{lingua franca} in science, on the labour market and in international politics (Die Zeit 2010).\pozn{English is undoubtedly used as the language of science in an overwhelming majority of key disciplines, i.e., in natural science, medicine and social sciences. A~survey performed by Amon (1991) shows that the use of other languages in scientific publications is rather marginal. German accounts for approximately one per cent of publications in the field of natural sciences, and approximately seven per cent in social sciences worldwide. German has traditionally played a~more important role only in selected humanities, such as Germanics, theology and philosophy. The significance of English on the labour market is a~logical outcome of the processes of internationalization and globalization of the world economy. It is worth mentioning that many international corporations domiciled in Germany also prefer English as their corporate language. As regards international politics, it seems to appropriate to stress that while German is one of the EU's working languages, its use is not particularly supported by German politics (Die Zeit 2010).} The following questions arise in this context:\\[-2mm]
\zacseznam
\item Given the importance of mutual economic relations with German-speaking countries, what is the level of German language proficiency of Czech university students, specifically, students of business and management, and thus their language skills when they seek positions on the national and international labour market? Business and management university graduates in particular can be expected to find management positions in companies, and to maintain intensive contacts with business partners on international markets as a~result.
\item What factors influence the level of German language proficiency of Czech university students of business and management?
\item How do Czech university students of business and management perceive the importance of German language proficiency in their careers and professional growth on the current Czech labour market, as compared to English?

\item How do Czech university students of business and management perceive the role of universities in improving German language proficiency?

\clearpage\newpage

\item How should business and management universities in the Czech Republic respond to student demand for greater German language proficiency in the form of courses, or at the level of individual subjects?
\konseznam

%\clearpage\newpage

      The above questions made it possible to formulate the main goal of the survey: to assess the German language proficiency of students at Czech business and management universities. Individual objectives can be derived from the main goal, specifically, to identify a) factors that had the greatest influence on the respondents' German language proficiency, b) the perception of the importance of the German language for professional development, and c) the perception of the role played by universities in language training.

      The following hypotheses were defined in light of the goals of the survey:
\zacseznam
\item[1.] There are statistically significant differences between the German language proficiency of the respondents and socio-demographic characteristics (i.e., age, course, form of study, work experience, intent to start one's own company, intent to work for an international company).
\item[2.] There are statistically significant differences between the perception of the importance of German for professional development in the current labour market in the Czech Republic and socio-demographic characteristics.
\item[3.] There are statistically significant differences between the perception of the role played by universities in the improvement of German language proficiency and socio-demographic characteristics.
\item[4.] There are statistically significant differences between the perception of the role played by universities in the improvement of German language proficiency and the respondents' interest in improving their language proficiency through the respective subjects (i.e., language training in general, specialized courses, specialized subjects taught in German).
\konseznam   
We assume that it is a~highly intriguing issue to map the state of German language proficiency in the Czech Republic almost one hundred years after the decline of the Austro-Hungarian Empire. Although ethnic, religious and language diversity dominated the monarchy, the German language created a~tie between its former countries. We also believe that our article represents a~contribution to the scope of European Association Comenius (EACO) i.e. ``the re-establishment of the academic and scientific ties among the countries of the former Danubian Monarchy''.%\\%[2mm]

\podnadpis{Literature Review}
One of the objectives of the employment policy of the European Union is to improve the language skills of citizens of its member states. In the European Commission's White Paper on Education and Training ``Teaching and Learning" (1996), multilingualism is referred to as a~typical feature of the identity of a~European citizen, and knowledge of multiple foreign languages is viewed as an important prerequisite for success on the labour market in a~knowledge-based economy. Baur (2009) deems this decision of the European Commission extraordinarily significant in that it is a~decision as to whether knowledge of European languages is to be restricted only to the territories of the individual states or whether it is to expand beyond their borders as well. ``This development directly relates to the creation of a~diverse and dynamic sense of belonging and awareness of European citizenship'' (\fntit{cittadinanza europea}).

Gudauner (2009) stresses the extraordinary importance of knowledge of foreign languages in the context of the Lisbon Strategy which heads of states and governments resolved at a~European Council meeting in Lisbon on 23--24 March, 2000. The Lisbon Strategy was a~response to the internationalization and globalization processes that lead \fntit{inter alia} to an ever closer integration of labour markets. At the Barcelona European Council meeting on 15--16 March, 2002, heads of states and governments agreed that language education in Europe was a~necessity from an early age. In the 2010 ``Work programme on the follow-up of the objectives of education and training systems in Europe", compiled in order to give effect to the conclusions of the Lisbon summit, the effort to improve foreign language proficiency within the EU was also emphasized.

In 2001, the Common European Framework of Reference for Language was applied; it classifies language knowledge in terms of six competence levels (from A1 to C2), and reflects the fact that ``the learning of a~language starts from a~complete lack of knowledge and progresses steadily towards autonomy''.

In 2003, the European Commission approved ``Promoting Language Learning and Linguistic Diversity: An Action Plan 2004--2006" in order to show all citizens the advantages of knowledge of foreign languages, to improve the approach to foreign language teaching in the educational system, and to create an environment recognizing linguistic diversity as a~typical feature of a~united Europe while seeing it as a~path to integration. The European Commission unequivocally states in the above-referenced document: ``Learning one \fntit{lingua franca} alone is not enough. Every European citizen should have meaningful communicative competence in at least two other languages in addition to his or her mother tongue." In the 2006 ``Recommendation of the European Parliament and of the Council on key competences for lifelong learning", the ability to communicate in foreign languages is listed as one of the eight key competences in lifelong learning.

The European Commission's 2007 Report on the implementation of the Action Plan ``Promoting language learning and linguistic diversity", which summarizes the outcome of the implementation of the Action Plan 2004--2006, analyzes the progress made by European countries in relation to those measures. It stresses in particular that the learning of foreign languages and multilingualism increased in importance on the political agendas of individual states. The slogan of the Barcelona summit, ``Mother tongue plus two other languages": making an early start, was applied in the respective support programmes of individual states, and ``a basic consensus was reached that the mastery of several languages was a~key competence of lifelong learning and that the quality of language training was extraordinarily important". Important and partly innovative approaches of individual member states consist for instance in introducing foreign language training to pre-school children, in the implementation of the CLIL model (Content and Language Integrated Learning), i.e., the teaching of specialized subjects in a~foreign language, in the expansion of foreign language training at secondary level and investment in teacher education. The report notes the following in the paragraph devoted to the situation in Belgium (the Flemish Community), the Czech Republic and Romania: ``In Belgium (Dutch-speaking Community), the Czech Republic and Romania some universities offer a~graduate programme in natural sciences, history or geography with a~minor in foreign languages, in some cases leading to a~double degree. This combination of different disciplines enhances language competences and should also prepare future CLIL teachers."

Baur (2009) is of the opinion that, from a~social and economic point of view, multilingualism is not only a~guarantee of flexibility in one's professional life but of higher social status as well. The European Commission recommends learning at least one language of a~neighbouring country because using it, respecting it and supporting it does not mean mere knowledge of another language, but also the building and strengthening of inter-cultural competence and peaceful coexistence. Broad ranks of the EU population further ought to possess basic English communication skills and sufficiently receptive language skills in at least one, if possible two, European languages. Bilingualism or multilingualism is not a~merely linguistic issue or a~matter of language didactics. It is further an issue of language policy, an issue of closeness to and distance from others, and an economic matter. H�ffe (1999) emphasizes that political integration requires the existence of a~European public. ``Its formation presupposes multilingualism because if Anglophiles only receive English and Francophiles French, the voices of other languages in Europe will remain unheard."

Baur (2009) warns of the risks inherent in the exclusive use of English as a~\fntit{lingua franca}. He warns against the temptation of following solely the path of bilingualism, i.e., the mother tongue + English, in educational policy. English as a~\fntit{lingua franca} is increasingly becoming a~medium of a~linguistic ``McDonaldization'' (Ritzer 2011) which is ``instrumentalized for the purposes of quick satisfaction of communication needs, and ultimately reduced from a~living organism to a~de-culturalized tool of information exchange''. Due to the growing involvement of foreign companies in various regions of Europe, as well as the growing importance of selected tourist destinations, the demand for other European languages in connection with inter-cultural competences, capability of teamwork and communication has been increasing on the European labour market. The processes of internationalization and globalization thus unequivocally require multilingualism, and this criterion is ever more important in decisions on HR issues. This trend clearly puts employees who speak several European languages at an advantage. 

P�rnbacher and Baur (2009) demonstrate with empirical data that foreign language knowledge is the third most importance competence following informatics and manufacturing and marketing and sales: ``Businesses wish to put their stakes on those employees who are able to communicate in languages important in terms of corporate strategy." Many markets in Europe open up through a~good knowledge of German, Italian and English. Although the position of English as a~lingua franca in international business is indisputable, it is appropriate to stress the fact that the other European languages mentioned above also occupy an indispensable position as languages of communication. German was indicated as the ``mainly" used language of communication by 52\% of respondents in Benelux countries, 24\% of respondents in Poland, 30\% of respondents in Croatia and Slovenia, and 14\% of respondents in the Czech Republic, Slovakia and Hungary. Italian was indicated as the ``mainly" used language of communication by 14\% of respondents in France and 23\% of respondents in Spain and Portugal. It may be assumed from the survey results that a~communication level knowledge of German and Italian also opens up markets to companies in Europe that are above and beyond Italy and German-speaking countries. This finding represents an assignment for politics and national educational systems that ought to place an even greater emphasis on language education and strengthening of inter-cultural competences.

In its study ``Effect on the European Economy of Shortages of Foreign Language Skills in Enterprise"/ELAN (2006), the British National Centre for Languages found that ``lack of language skills in companies causes loss of prospective business, and may be a~barrier to entry onto various European markets." The study involved 2,000 SMEs from all over Europe, whose data was confronted, using correlation analysis, with data obtained from 30 multinational holdings and information from experts from the countries involved. The study shows that, from the employees' perspective, adequate language skills are an increasingly important factor in their search for satisfactory posts on the labour market. The ELAN study further shows that nearly half of the SMEs taking part in the survey plans to expand to new foreign markets, and expects increased requirements on the language skills of their employees. Despite this fact, the companies do not invest in language training on an individual basis: instead, they tend to expect that the national systems of education would provide them with adequately qualified employees. The study points out that this strategy is unfortunate. The study proposes that direct investment in the improvement of language skills be increased, as they can be expected to have a~positive effect on the corporate economy, in particular at the level of productivity and export development.

It is evident that all European educational systems, as well as further education models of enterprises, are facing the increasing importance of learning foreign languages, since ``there is a~dire need to educate multilingual and multicultural citizens in a~context where the linguistic consequences of globalization are more and more evident" (Lasagabaster 2008). 

The extent of knowledge of selected foreign languages in individual European countries and the perception of their importance is suggested by the outcome of the Eurobarometer 386 (2012) and Eurobarometer 243 (2006) surveys commissioned by the European Commission. The overall objective of these surveys was to understand European citizens' experiences and perceptions of multilingualism. Fieldwork was completed in 2005 and 2012. These macro-surveys involved 28,694 and 26,751 respondents, respectively.

The survey results (2012) showed that, while just over half of Europeans (54\%) are able to hold a~conversation in at least one additional language, a~quarter (25\%) is able to speak at least two additional languages and one in ten (10\%) is conversant in at least three. There were significant differences between countries. Almost all respondents in Luxembourg (98\%), Latvia (95\%), the Netherlands (94\%), Malta (93\%), Slovenia and Lithuania (92\% each), and Sweden (91\%) say that they are able to speak at least one language in addition to their mother tongue. On the other hand, countries where respondents are least likely to be able to speak any foreign language are Hungary (65\%), Italy (62\%), the UK and Portugal (61\% in each), and Ireland (60\%).

The five most widely spoken foreign languages remain English (38\%), French (12\%), German (11\%), Spanish (7\%) and Russian (5\%). At a~national level English is the most widely spoken foreign language in 19 of the 25 Member States where it is not an official language (i.e. excluding the UK and Ireland). Two-thirds of Europeans (67\%) consider English one of the two most useful languages. Those languages perceived as the next most useful are the following: German (17\%), French (16\%), Spanish (14\%) and Chinese (6\%). 

In comparison to 2005 (Special Eurobarometer 243 2006), the proportion of respondents able to speak at least one foreign language has decreased notably in the Czech Republic ($-$12 points to 49\%). Also, in other countries there has been a~similar downward shift since 2005: in Slovakia ($-$17 percentage points to 80\%), Bulgaria ($-$11 points to 48\%), Poland ($-$7 points to 50\%), and Hungary ($-$7 points to 35\%). Within these countries the proportions of respondents able to speak foreign languages such as Russian and German have declined as well. For example, the proportions speaking German are down in the Czech Republic ($-$13 points to 15\%), Slovakia ($-$10 points to 22\%) and Hungary ($-$7\,points to 18\%). 

Furthermore, respondents were asked to name the two languages, apart from their mother tongue, that they believed to be most useful for their personal development. Two-thirds of Europeans (67\%) think that English is one of the two most useful languages. Less than one in five Europeans mention German (17\%), French (16\%) and Spanish (14\%). Compared with the results from the 2005 survey, there has been a~notable decrease in the proportion thinking that French is important ($-$9 points), and a~somewhat smaller decrease in those thinking German is an important language for personal development ($-$5 points). The most significant worsening of opinion on the usefulness of German is in the Czech Republic ($-$23 points to 32\%), Slovakia ($-$17 points to 44\%), Lithuania ($-$15 points to 13\%), Poland ($-$15 points do 30\%), in the UK ($-$15 points to 12\%), Bulgaria ($-$14 points to 20\%) and Luxembourg ($-$13 points to 47\%). On the other hand, the view that German is one of the most useful languages for personal development is most widespread in Slovenia (50\%), Denmark and Hungary (48\% in each), Luxembourg (47\%), and the Netherlands and Slovakia (44\% in each).

In 2010, the Institute for Social and Economic Analyses, in collaboration with the Deutsch-Tschechische Zukunftsfonds and Goethe-Institute Prag, carried out a~survey focusing on the foreign language proficiency of the Czech population, in particular its ability to communicate in German. The survey focused on five languages: English, German, Russian, French and Spanish. The data set is representative of the Czech population over 15 years of age. The total number of respondents in the set is 1,061 (isea 2010).

The outcome of the survey suggests that the ``ability and willingness to learn foreign languages is not great in the Czech Republic" (isea 2010), as 54\% of the respondents are unable to communicate ``well" in any of the languages concerned, 27\% are conversant in one of them, 14\% in two languages, and 5\% of the respondents are conversant in three or more of the languages. The fact that 23\% of university-educated Czechs are unable to communicate well in any of the languages concerned is a~matter of concern. The survey results further indicate that, while 27\% of the respondents state they are able to communicate ``quite well" in English, less than 10\% of the respondents claim they speak English ``very well" or ``have no problem with both oral and written communication". German ranks second: 22\% of the respondents state they can communicate in German ``quite well", but less than 5\% of the respondents claim to be able to communicate in German ``very well" or ``have no problem with both oral and written communication". The authors of the survey stress that while German is not the number one world language for Czechs, it ``unequivocally" enjoys the status of the number two language.

The survey further examined the perception of the importance of German in career success. 77\% of the respondents agreed with the claim that ``German is useful for career success". The study concludes in this context that ``while German ranks second after English in terms of importance for life, which is not a~surprising conclusion, it is nonetheless a~much more important language than the other world languages, probably also thanks to its perceived importance for one's career. This potential is much higher in the university-educated population."

The outcome of the survey stresses the key importance of schools in the process of foreign language proficiency acquisition: ``More than one-half of the Czech population (54\%) states that their current language skills were strongly influenced by language learning opportunities at school." The authors of the study thus recommend in conclusion that ``more favourable conditions for a~greater expansion of foreign language proficiency in at least two world languages be created at schools, in particular with a~view to the Czech Republic's geographic location, and the importance of mastery of world languages for nations of the size of the Czech nation".\\%[2mm]

\podnadpis{Methodology}
An approach based on statistical testing of hypotheses was used to attain the goal of the survey. Collection of data using a~standardized questionnaire took place in October to November 2011. The respondents were students of two Czech business and management universities -- one public and one private -- in the South Moravian Region. Given the scope of both universities, however, the surveyed sample can be expected to include respondents from all regions of the Czech Republic.

The first part of the questionnaire examines the German language proficiency of the respondents (based on self-evaluation), the reasons for low or non-existent German language proficiency, as well as factors that influenced the fact that the respondent is conversant in German. The second part of the questionnaire focused on the perception of the importance of German for professional development on the current labour market in the Czech Republic (as compared to English). In the third part of the questionnaire, the respondents comment on their perception of the university's role in the improvement of their German language proficiency and what subjects they would be interested in in order to improve their German language proficiency. The fifth part of the questionnaire examines the socio-demographic characteristics of a~selected sample of respondents.

% puvodne Education/ MEYS (2011)
A total of 257 questionnaires ($N=257$) in hard copy form was collected. According to data from the Ministry of Education, Youth and Physical Education (MEYS 2011), there was a~total of 96,500 students enrolled in business and management programmes at public and private universities in the Czech Republic in 2011. The share of the set of respondents in the basic set thus amounts to 0.27\%.\\

%\clearpage\newpage

\oddil{Composition of the sample surveyed}
The composition of the sample surveyed is shown in Table 1. Women (57\%) outnumber men (39\%). The age group of 18--25 (68\%) is dominant, respondents aged 26--55 accounted for 27.24\% in aggregate. As regards the type of study, full-time students (56\%) outnumber students enrolled in a~combined form of study (38\%), and as regards the study programme, students enrolled in Bachelor's programmes (56\%) outnumbered students enrolled in Master's programmes (40\%).

The composition of the sample surveyed can be compared to certain statistical data concern\-ing the basic set of students of all public and private universities in the Czech Republic in business and management programmes in 2011.\pozn{I.e., natural sciences, technical sciences, agricultural-forestry sciences and veterinary sciences, medical and pharmaceutical sciences, art sciences, social sciences, services.} MEYS (2011) states that students enrolled in Bachelor's programmes account for 69\% of the total number of students, and students enrolled in Master's programmes for 29\% of all students. Full-time students account for 66\%, while 34\% are students enrolled in extramural (combined) study. Women represent nearly 56\% of all students enrolled in all the study programmes in the Czech Republic. In terms of age make-up, the age group of 18--25 is dominant (nearly 72\% of the total population), while the other agreement groups account for a~significantly smaller percentage (28\%). It can be noted that the survey sample of respondents is comparable to the basic set because of a~similar distribution of features (age, sex, study programme and type).

The work experience of the respondents was also surveyed. More than three-quarters indicate they have work experience, as opposed to some 19\% of students who have no experience as of yet. Given the distribution of the survey sample, a~large percentage of students enrolled in full-time study thus already has some work experience. The fields in which the respondents would like to work after graduation are dominated by ``accounting, taxes and corporate finance", ``business management", ``trade" and ``human resources". 46\% of the respondents declare they wish to start their own business, i.e., slightly fewer than those who see their future as employees (48\%). As regards the respondents' attitude to foreign language skills, information concerning their intent of working for an international company is presumably also significant: over 75\% of the respondents declare that that is what they plan to do.

\clearpage\newpage

\popis{Table 1: Socio-demographic characteristics of the sample of respondents}
\vspace*{-4mm}\tabcolsep 7pt
\footnotesize\zacltab{|l|c|c|c|c|}\hline
			&\bf	Frequency 	&\bf	Cumulative  	&\bf	Relative 		&\bf	Cumulative relative\\
			&				&\bf	frequency		&\bf	frequency (\%)	&\bf	 frequency (\%)	\\\hline
\endhead
\mc{5}{r}{\fntit{Continued on next page}}	\endfoot
\endlastfoot
\mc{5}{|l|}{\bf Sex:}	\\\hline
male			&99&99&38.52&38.52\\\hline
female		&146&245&56.81&95.33\\\hline
n/a 			&12&257&4.67&100.00\\\hline
\mc{5}{|l|}{\bf Age:}	\\\hline
18--25		&175&175&68.09&68.09\\\hline
26--35		&39&214&15.18&83.27\\\hline 			%\newpage
36--45		&25&239&9.73&93.00\\\hline
46--55		&6&245&2.33&95.33\\\hline
n/a 			&12&257&4.67&100.00\\\hline
\mc{5}{|l|}{\bf Programme:}	\\\hline
Bachelor's		&143&143&55.64&55.64\\\hline
Master's 		&102&245&39.69&95.33\\\hline
n/a 			&12&257&4.67&100.00\\\hline
\mc{5}{|l|}{\bf Type of study:}	\\\hline
full-time 		&144&144&56.03&56.03\\\hline
Combined 	&98&242&38.13&94.16\\\hline
n/a 			&15&257&5.84&100.00\\\hline
\mc{5}{|l|}{\bf Intended field of work after graduation}	\\\hline
finance, accounting and taxes 	&63&63&24.51&24.51\\\hline
business management 		&6&69&2.33&26.84\\\hline
HR 	&5&74&1.95&28.79\\\hline
logistics	&1&75&0.39&29.18\\\hline
banking 	&4&79&1.56&30.74\\\hline
trade 	&6&85&2.33&33.07\\\hline
industry 	&4&89&1.56&34.63\\\hline
services 	&2&91&0.78&35.41\\\hline
health care 	&2&93&0.78&36.19\\\hline
education 	&2&95&0.78&36.97\\\hline
marketing 	&4&99&1.56&38.53\\\hline
business 	&3&102&1.17&39.70\\\hline
telecommunications 	&1&103&0.39&40.09\\\hline
sports 	&1&104&0.39&40.48\\\hline
other (unspecified) 	&21&125&8.17&48.65\\\hline
n/a 	&132&257&51.35&100.00\\\hline
\mc{5}{|l|}{\bf Work experience}\\\hline
yes 	&197&197&76.65&76.65\\\hline
no 	&48&245&18.68&95.33\\\hline
n/a 	&12&257&4.67&100.00\\\hline
\mc{5}{|l|}{\bf Wish to start one's one business}	\\\hline
Yes 	&119&119&46.30&46.30\\\hline
No 	&123&242&47.86&94.16\\\hline
n/a 	&15&257&5.84&100.00\\\hline
\mc{5}{|l|}{\bf Wish to work in an international company}	\\\hline
yes 	&193&194&75.10&75.10\\\hline
no 	&50&243&19.46&94.56\\\hline
n/a 	&14&257&5.44&100.00\\\hline
\konltab

\oddil{Statistical processing of data}
The data gained from the survey was processed in the following consecutive steps: 1)~Question\-naire data were entered into a~database for statistical processing; 2) Descriptive analysis of all the parameters studied was performed; 3) An analysis of contingency tables and their statistical evaluation using the M-V chi-square test were performed.

Descriptive statistics and its commentary are based on real numbers. Statistical processing of source data is provided in Tables 1 through 13. In the part dedicated to results, only data to which answers had been provided is commented on (i.e., the sum of relative frequencies is not 100\%).

The data was evaluated at the significance level of $\alpha=5\%$. The entire statistical evaluation was performed by Statistica.CZ, Version 9.\\

\podnadpis{Survey Results}
\oddil{Respondents' German language proficiency}
The first part of the survey focused on assessing the respondents' ability to communicate in German. The respondent was asked to assess, using the self-evaluation scale under the Common European Framework of Reference for Language, his/her level of comprehension of the German language in terms of listening comprehension, reading, the ability to converse and the quality of written output. The survey results (see Table 2) indicate that over 76\% of the respondents claim to be able to communicate in German (categories A1--C2). However, considering the fact that respondents at levels B2--C2 are able to communicate in German very well, it is apparent that a~mere 7.41\% of the respondents is able to communicate orally and in writing without any problem.

\popis{Table 2: Respondent's ability to communicate in German}
\tabcolsep 7pt
\footnotesize\zactab{|l|c|c|c|c|}\hline
\bf Category 		&	\mc{4}{|c|}{\bf Frequency table}	\\\cline{2-5}
				&\bf	Frequency 	&\bf	Cumulative 	&\bf	Relative 		&\bf	Cumulative relative \\
				&				&\bf	frequency 		&\bf	frequency (\%)	&\bf	frequency (\%)\\\hline
I cannot speak German at all 	&59&59&22.96&22.96\\\hline
A1 -- beginner 	&70&129&27.24&50.20\\\hline
A2 -- elementary 	&60&189&23.35&73.55\\\hline
B1 -- intermediate 	&47&236&18.29&91.84\\\hline
B2 -- upper intermediate 	&13&249&5.06&96.90\\\hline
C1 -- advanced 	&5&254&1.95&98.85\\\hline
C2 -- proficient 	&1&255&0.39&99.24\\\hline
n/a 	&2&257&0.76&100.00\\\hline
\kontab\\[4mm]

The results of the questionnaire survey can be compared to MEYS's statistical data (2011). According to the latter, a~foreign language is studied at university by less than one-half of university students in the Czech Republic, regardless of their study programme (see Table 3). This is a~rather low figure which does not suggest that Czech students are particularly willing to learn foreign languages. Of those who do study foreign languages, most of them study English. The position of English has been stable over the last five years (a~73\%\,share). The study of German ranks second although its position has weakened slightly. As compared to the original 25\% in 2007, less than 20\% of students studied German in 2011. Nonetheless, the position of German as the second most frequently studied language in the Czech Republic remains incontestable. The shares of other languages are significantly lower. It also needs to be pointed out that the number of students studying two or more foreign languages is declining. In 2007, it accounted for 25\%, in 2011, for 17\%.

%\clearpage\newpage

\popis{Table 3: Teaching of foreign languages at universities in all study programmes in the Czech Republic (2011)}
\tabcolsep 7.2pt
\footnotesize\zactab{|l|c|c|c|c|c|}\hline
\bf Year 		&\bf	2007	&\bf	2008	&\bf	2009	&\bf	2010	&\bf	2011\\\hline
University students 	in total 	&343,990 &368,073 &389,066 &396,073 &392,429 \\\hline
No. of students studying foreign languages 	&&&&&\\
 in total 	&159,445 &166 490 &172,330 &172,243 &177,941 \\\hline
\mc{6}{|l|}{\fntit{of that:	}}\\\hline
1 language 	&119,918 &128,015 &133,002 &136,315 &147,027 \\\hline
2 languages	&36,008 &34,485 &36,092 &33,071 &28,956 \\\hline
3 and more languages 	&3,519 &3,990 &3,236 &2,857 &1,958 \\\hline
\mc{6}{|l|}{\fntit{of that, the number of students studying:}}\\\hline
English 	&116,574 &122,408 &127,175 &127,249 &129,480 \\\hline
French 	&9,601 &9,494 &9,895 &9,289 &8,880\\\hline
German 	&39,401 &38,819 &38,390 &35,775 &35,107 \\\hline
Russian 	&10,874 &10,640 &11,677 &10,907 &10,103 \\\hline
Spanish 	&8,903 &10,154 &11,074 &10,341 &9,418 \\\hline
Italian 	&1,952 &3,133 &2,421 &2,219 &2,095 \\\hline
Latin 	&8,878 &9,603 &9,335 &9,611 &9,297 \\\hline
other European language 	&2,434 &2,737 &2,048 &1,943 &1,408 \\\hline
other language 	&1,167 &1,322 &1,182 &1,104 &1,262 \\\hline
\mc{6}{|l|}{\fntit{Share of students:}}\\\hline
studying a~foreign language in the total &&&&&\\
number of university students	&46.35\%&45.23\%&44.29\%&43.49\%&45.34\%\\\hline
of German in the total number	&&&&&\\
of language students 	&24.71\%&23.32\%&22.28\%&20.77\%&19.73\%\\\hline
of English in the total number 	&&&&&\\
of language students 		&73.11\%&73.52\%&73.80\%&73.88\%&72.77\%\\\hline
studying only one foreign language 	&75.21\%&76.89\%&77.18\%&79.14\%&82.63\%\\\hline
studying two or more foreign languages 	&24.79\%&23.11\%&22.82\%&20.86\%&17.37\%\\\hline
\kontab
\zdroj{Source: MEYS (2011)}\\[-4mm]
{\small\fntit{Note: The number provided is that of students learning a~language regardless of how many languages they are studying.}}\\[-2mm]

As in the \fntit{isea} survey (2010), the questionnaire included a~question designed to identify the main factors that influenced the fact that the respondent is able to communicate in German. Respondents were able to check multiple answers and could choose from the following:
\zacseznam
\item[a)]	It was my parents' wish;
\item[b)]	I had the opportunity to study the language at school;
\item[c)]	There was no other choice at the school I~attended;
\item[d)]	Motivating environment for study of the language at school (teachers, schoolmates);
\item[e)]	The necessity to speak a~language for the purposes of employment or career success;
\item[f)]	My relationship to the country and culture associated with the language;
\item[g)]	School exchange or study stay abroad;
\item[h)]	Work stay abroad;
\item[i)]	Friends or relatives in German-speaking countries.
\konseznam

%\clearpage\newpage

%%%%%%%%%%% upravit
            65\% of the respondents answered the question. The survey results (see Table 4) stress the role of school in the teaching of the German language. More than 13\% of the respondents state that their ability to communicate in German was influenced by the school environment which gave them an opportunity to learn the language. 14\% of the respondents stated that they learned German because the school did not offer any language other than German (i.e., German was taught as a~compulsory subject in this case). Where respondents checked multiple answers (29\%), a~combination of two factors prevails in an overwhelming majority of cases: the school environment (opportunity to study, compulsory subject, motivation to study the language) and the necessity of knowing the language for work purposes. Factors such as future career contemplations, parents' wishes, relationship to the country and culture or a~study or work stay play a~marginal role in the acquisition of German language proficiency.

\popis{Table 4: Factor that influenced the fact that the respondent can communicate in German}
\tabcolsep 6.2pt
\footnotesize\zactab{|l|c|c|c|c|}\hline
\bf Category 		&	\mc{4}{|c|}{\bf Frequency table}	\\\cline{2-5}
				&\bf	Frequency 	&\bf	Cumulative 	&\bf	Relative 		&\bf	Cumulative relative \\
				&				&\bf	frequency 		&\bf	frequency (\%)	&\bf	frequency (\%)\\\hline
parents' wish 	&3&3&1.17&1.17\\\hline
opportunity to learn the  	&&&&\\
language at school 	&34&37&13.23&14.40\\\hline
no other option 	&36&73&14.01&28.41\\\hline
motivating school environment 	&4&77&1.56&29.97\\\hline
profession 	&8&85&3.11&33.08\\\hline
relationship to the country &&&&\\
and culture &3&88&1.17&34.25\\\hline
school exchange, study stay &1&89&0.39&34.64\\\hline
work stay 	&2&91&0.78&35.42\\\hline
friends, relatives 	&2&93&0.78&36.20\\\hline
multiple answers 	&75&168&29.18&65.38\\\hline
n/a 	&89&257&34.62&100.00\\\hline
\kontab\\[4mm]

If we examine the reasons for low or non-existent German language skills (in the case of respondents who either do not speak German at all or are at A1 level), the survey results once again indicate that schools play a~key role in the teaching of German. The respondent was asked to indicate the main reason and was able to check only one of the following answers:
\clearpage\newpage
\vspace*{-6mm}
\zacseznam
\item[a)]	German was not taught at the schools I~attended;
\item[b)]	I learned German a~little but have forgotten it;
\item[c)]	I did not want to learn German because it is a~very difficult language;
\item[d)]	I did not want to learn German because I~do not think I~will need it for my life or career;
\item[e)]	I did not want to learn German because I~do not like it as a~language;
\item[f)]	I did not want to learn German because of another reason. Please indicate why;
\item[g)]	I have no gift for languages and so I~do not study them;
\item[h)]	I am only able to learn one foreign language properly.
\konseznam

%\clearpage\newpage

            Table 5 indicates that there are two dominant reasons for low or non-existent German language skills which relate to the school environment: the first being the fact that the respondent did study German but has forgotten it (23\%), the second the inability to learn the language at school (8\%). Those who did not want to learn German indicate they do not like it as a~language as the main reason (13\%). Other reasons why the respondents did not want to learn German include lack of use for it, difficult grammar, preference for English as the sole foreign language, or preference of French, Spanish or Russian as the second foreign language.

\popis{Table 5: Main reasons for low or non-existent German language skills}
\tabcolsep 6.4pt
\footnotesize\zactab{|l|c|c|c|c|}\hline
\bf Category 		&	\mc{4}{|c|}{\bf Frequency table}	\\\cline{2-5}
				&\bf	Frequency 	&\bf	Cumulative 	&\bf	Relative 		&\bf	Cumulative relative \\
				&				&\bf	frequency 		&\bf	frequency (\%)	&\bf	frequency (\%)\\\hline
German was not taught 	&&&&\\
at schools 			&21&21&8.17&8.17\\\hline
German was taught but 	&&&&\\
I~have forgotten it 		&54&75&21.01&29.18\\\hline
I did not want to learn German  	&&&&\\
-- it is a~very difficult language 	&1&76&0.39&29.57\\\hline
I did not want to learn German 	&&&&\\
-- no need for my career 	&3&79&1.17&30.74\\\hline
I did not want to learn German 	&&&&\\
-- I~don't like it 	&30&109&11.67&42.41\\\hline
I did not want to learn German 	&&&&\\
-- other reason 		&17&126&6.61&49.02\\\hline
I have no gift for languages 	&0&0&0.00&0.00\\\hline
I am only able to learn one 	&&&&\\
foreign language 	&2&128&0.78&49.80\\\hline
n/a 		&129&257&50.20&100.00\\\hline
\kontab\\[2mm]

Further, the following statistically significant differences between the German language proficiency of the respondents and the individual components of socio-demographic characteristics were examined. The authors tested the existence of the following relations:\\[-4mm]
\zacseznam
\item[a)] Is there a~relationship between the respondent's age and his/her ability to communicate in German?
\item[b)] Is there a~relationship between the respondent's study programme and his/her ability to communicate in German?
\item[c)] Is there a~relationship between the respondent's type of study and his/her ability to communicate in German?
\item[d)] Respondents with work experience are able to communicate in German to a~greater extent.
\item[e)] Respondents who wish to start their own business are able to communicate in German to a~greater extent.
\item[f)] Respondents who wish to work for an international company are able to communicate in German to a~greater extent.
\konseznam
            No statistically significant difference was found at the level of the individual components of socio-demographic characteristics (data not provided).\\
      
     %\clearpage\newpage 
      

\oddil{Perception of the importance of German for career success\\and professional growth}
In a~further part of the survey, we asked about the respondents' perception of the importance of German for their career success and professional growth on the current labour market in the Czech Republic. Nearly 70\% of the respondents indicated that German was important for success on the labour market; of that, 57\% indicated it was quite important and 12\% that it was very important. Only 2.33\% of the respondents believe that German is completely unimportant for work success and professional growth (see Table 6).
The respondents were further asked to what extent they agree with the opinion that knowledge of one foreign language -- English -- is sufficient for success and professional growth on the current labour market in the Czech Republic (see Table 7). Nearly 48\% definitely or slightly agree with this opinion, while 50\% definitely disagree. 

The assessment of the responses to both questions outlined above shows that the perception of the importance of German for career and profession is high, despite the fact that most respondents believe that knowledge of English is currently sufficient for success on the labour market in the Czech Republic. This finding once again confirms the significant potential of German as the second foreign language.

\popis{Table 6: Perception of the importance of German for work success and professional growth on the current labour market in the Czech Republic}
\tabcolsep 5.4pt
\footnotesize\zactab{|l|c|c|c|c|}\hline
\bf Category 		&	\mc{4}{|c|}{\bf Frequency table}	\\\cline{2-5}
				&\bf	Frequency 	&\bf	Cumulative 	&\bf	Relative 		&\bf	Cumulative relative \\
				&				&\bf	frequency 		&\bf	frequency (\%)	&\bf	frequency (\%)\\\hline
knowledge of German completely	&&&&\\
unimportant 	&6&6&2.33&2.33\\\hline
knowledge of German rather 	&&&&\\
unimportant 	&69&75&26.85&29.18\\\hline
knowledge of German rather 	&&&&\\
important 	&146&221&56.81&85.99\\\hline
knowledge of German very 	&&&&\\
important 	&31&252&12.06&98.05\\\hline
n/a 	&5 &257&1.95&100.00\\\hline
\kontab

\clearpage\newpage

\popis{Table 7:  Identification with the view that knowledge of one language -- English -- is sufficient for success and professional growth on the labour market in the Czech Republic}
\tabcolsep 10.4pt
\footnotesize\zactab{|l|c|c|c|c|}\hline
\bf Category 		&	\mc{4}{|c|}{\bf Frequency table}	\\\cline{2-5}
				&\bf	Frequency 	&\bf	Cumulative 	&\bf	Relative 		&\bf	Cumulative relative \\
				&				&\bf	frequency 		&\bf	frequency (\%)	&\bf	frequency (\%)\\\hline
definitely disagree &29&29&11.28&11.28\\\hline
slightly disagree 	&99&128&38.52&49.80\\\hline
slightly agree 	&93&221&36.19&85.99\\\hline
definitely agree 	&30&251&11.67&97.66\\\hline
n/a 	&6&257&2.34&100.0000\\\hline
\kontab\\[2mm]

Statistically significant differences between the perception of the importance of German for professional growth on the current labour market in the Czech Republic and the individual components of socio-demographic characteristics were also examined. The following statistically significant differences were found to exist:

{\leftskip 5mm
      a) A~statistically significant difference was found to exist between the following components: {\bf respondent's age} and {\bf the perception of the importance of German} for success and professional growth on the current labour market in the Czech Republic ($\chi^2 = 16.79$; $df=9$; $p=0.049$).\\[-2mm]\par}

German is considered important (rather or very important) for professional life by 128 respondents in the age group of 18--25 (i.e., 74\% within that age group), 19 respondents in the age group of 26--35  (i.e., 50\% within that age group), 20 respondents in the age group of 36--45 (i.e., 83\% within that age group), and 4 respondents in the age group of 46--55 (i.e., 67\% within that age group). The results confirm that there is a~statistically significant difference between the respondent's age and his/her perception of the importance of German. While in the agreement groups of 18--25 and 36--45, German is perceived as an unimportant language by a~minority of the respondents, in the agreement group of 26--35, German is perceived as an unimportant language by one-half of the respondents; on the other hand, not a~single respondent in this agreement group expressed the opinion that German was a~completely unimportant language in terms of success on the domestic labour market. The opinion that German is completely unimportant for professional purposes is held by an insignificant number of respondents across all the age groups (see Table 8).
      
\popis{Table 8: Contingency table -- relationship between the respondent's age and his/her perception of the importance of German for success and professional growth on the labour market in the Czech Republic}
\tabcolsep 6.8pt
\footnotesize\zactab{|c|c|c|c|c|c|c|}\hline
\bf Age		&\bf	Frequency 	&	\mc{5}{|c|}{\bf Perception of the importance of German}	\\\cline{3-7}
			&				&	German 		&	German		&	German 	&	German 	& \\
			&				&	completely	&	rather		&	rather	&	very		&	sum	\\
			&				&	unimportant 	&	unimportant 	&	important 	&	important 	&	\\\hline
18--25 	&	absolute (relative) 	&4 (2\%)&41 (17\%)&105 (44\%)&23 (10\%)&173 (72\%)\\\hline
26--35 	&	absolute (relative)	&0 (0\%)&19 (8\%)&15 (6\%)&4 (2\%)&38 (16\%)\\\hline
36--45 	&	absolute (relative) 	&0 (0\%)&4 (2\%)&17 (7\%)&3 (1\%)&24 (10\%)\\\hline
46--55 	&	absolute (relative)	&1 (0\%)&1 (0\%)&3 (1\%)&1 (0\%)&6 (2\%)\\\hline
total 		&	absolute (relative)	&5 (2\%)&65 (27\%)&140 (58\%)&31 (13\%)&241 (100\%)\\\hline
\kontab

{\small\fntit{Note: The information on relative number is based on the total number of answers ($N=241$).}}

\clearpage\newpage

{\leftskip 5mm
      b) A~statistically significant difference was found to exist between the following components: {\bf respondent's work experience} and {\bf the perception of the importance of German} for success and professional growth on the labour market in the Czech Republic ($\chi^2 = 13.55$; $df=3$; $p<0.05$).\\\par}

In both categories of respondents (with and without work experience), the opinion that German is rather or very important for success on the Czech labour market is dominant. In the category of respondents with work experience, this opinion is prevailing in 137\,respondents (i.e., 71\% in this category), in the category of respondents without work experience, 34 respondents hold this view (i.e., 72\% in this category). However, if we compare the frequency of positive answers in the answer variant ``German is very important", we find that while 30 respondents with work experience (i.e., 15\% in this category) consider German to be a~very important language, only one person deems German to be a~very important language in the group of respondents without work experience (i.e., 2\% in this category). If we disregard those respondents who consider German completely unimportant in terms of work success, we arrive at the conclusion that 98\% of all the respondents (both with or without work experience) perceive German as a~language important for professional purposes, although there are differences in the perception of the degree of its importance between the two categories (see Table 9).

\popis{Table 9: Contingency table -- relationship between the respondent's work experience and his/her perception of the importance of German for success and professional growth on the labour market in the Czech Republic}
\tabcolsep 5.4pt
\footnotesize\zactab{|c|c|c|c|c|c|c|}\hline
\bf Work		&\bf	Frequency 	&	\mc{5}{|c|}{\bf Perception of the importance of German}	\\\cline{3-7}
\bf Experience	&				&	German 		&	German		&	German 	&	German 	& \\
			&				&	completely	&	rather		&	rather	&	very		&	sum	\\
			&				&	unimportant 	&	unimportant 	&	important 	&	important 	&	\\\hline
no 	&absolute (relative)&3 (1\%)&10 (4\%)&33 (14\%)&1 (0\%)&47 (20\%)\\\hline
yes 	&absolute (relative)&2 (1\%)&55 (23\%)&107 (44\%)&30 (12\%)&194 (80\%)\\\hline
total 	&absolute (relative)&5 (2\%)&65 (27\%)&140 (58\%)&31 (13\%)&241 (100\%)\\\hline
\kontab

{\small\fntit{Note: The information on relative number is based on the total number of answers ($N=241$).}}\\

{\leftskip 5mm
      c) A~statistically significant difference was found to exist between the following components: {\bf knowledge of English is sufficient for success and professional growth on the current labour market in the Czech Republic} and {\bf the perception of the importance of German} ($\chi^2 = 7.43$; $df=3$; $p=0.048$).\\\par}

Table 10 shows that in the category of respondents who perceive German as a~very important language for career success and professional growth (31 persons), 29\% (i.e., 9\,persons) definitely disagree with the view that the knowledge of English is sufficient. On the other hand, in the category of respondents who perceive German as a~rather important language (146 persons), 11\% (i.e., 16 persons) definitely disagree with this opinion. In both categories under examination, a~comparable relative frequency of respondents who checked the ``slightly disagree" option can be observed (49\% of the respondents perceiving German as rather important and 48\% of the respondents perceiving German as very important).\\[-4mm]

\popis{Table 10: Contingency table -- relationship between agreement with the opinion that knowledge of English is sufficient, and the perception of the importance of German for success and professional growth on the labour market in the Czech Republic}
\tabcolsep 4.2pt
\footnotesize\zactab{|c|c|c|c|c|c|c|}\hline
\bf Perception of	&\bf	Frequency 	&	\mc{5}{|c|}{\bf Agreement with the Opinion that the Knowledge}	\\
\bf the Importance	&				&	\mc{5}{|c|}{\bf  of English is Sufficient} 	\\\cline{3-7}
\bf of German		&				&	definitely 		&	slightly		&	slightly 	&	definitely 	&  sum\\
				&				&	disagree		&	disagree		&	agree	&	agree	&	\\\hline
German rather important 	&absolute (relative)&16 (9\%)&72 (41\%)&45 (25\%)&13 (7\%)&146 (82\%)\\\hline
German very important 	&absolute (relative)&9 (5\%)&15 (8\%)&6 (3\%)&1 (1\%)&31 (18\%)\\\hline
total 	&absolute (relative)&25 (14\%)&87 (49\%)&51 (29\%)&14 (8\%)&177 (100\%)\\\hline
\kontab

{\small\fntit{Note: The information on relative number is based on the total number of answers ($N=177$).}}\\

%\clearpage\newpage

{\leftskip 5mm
      d) A~statistically significant difference was found to exist between the following components: {\bf knowledge of English is sufficient for success and professional growth on the current labour market in the Czech Republic} and {\bf the type of study} ($\chi^2=7.53$; $df=3$; $p=0.047$).\\\par}

Table 11 suggests that while full-time students mostly do not agree with the opinion that knowledge of English is sufficient for professional success (83 persons, or 59\%, opted for ``definitely disagree" or ``slightly disagree"), 58\% of the students enrolled in a~combined type of study (56 persons) hold this view. This is a~significant finding with a~view to the fact that students enrolled in a~combined type of study can be expected to have closer links to corporate practice, and as such possess a~more thorough knowledge of employers' requirements. The importance of German for professional success is completely denied by 17\% of students enrolled in a~combined type of study (16 persons) and 9\% of full-time students (13 persons). As for the opinion that knowledge of English is sufficient for professional success, 13\% of full-time students (18 persons) and 10\% of students enrolled in a~combined type of study (10 persons) definitely disagree with this opinion, respectively.

\popis{Table\;11: Contingency table -- the relationship between agreement with the opinion that knowledge of English is sufficient, and type of study}
\tabcolsep 6.5pt
\footnotesize\zactab{|c|c|c|c|c|c|c|}\hline
\bf Type of study	&\bf	Frequency 	&	\mc{5}{|c|}{\bf Agreement with the Opinion that the Knowledge}	\\
				&				&	\mc{5}{|c|}{\bf  of English is Sufficient} 	\\\cline{3-7}
				&				&	definitely 		&	slightly		&	slightly 	&	definitely 	&  sum\\
				&				&	disagree		&	disagree		&	agree	&	agree	&	\\\hline
full-time &absolute (relative)&18 (8\%)&65 (27\%)&45 (19\%)&13 (5\%)&141 (59\%)\\\hline
combined &absolute (relative)&10 (4\%)&30 (13\%)&40 (17\%)&16 (7\%)&96 (41\%)\\\hline
total 	&absolute (relative)&28 (12\%)&95 (40\%)&85 (36\%)&29 (12\%)&237 (100\%)\\\hline
\kontab

{\small\fntit{Note: The information on relative number is based on the total number of answers ($N=237$).}}\\

\oddil{The role of university in the improvement of German language proficiency}

The last part of the questionnaire focused on mapping the role played by the university in the improvement of German language proficiency. We further examined the question of how business and management universities in the Czech Republic ought to respond to potential demand by students for improvement of German language skills at the level of individual subjects.

The results of the survey indicate (see Table 12) that 65\% of respondents believe that a~business and management university should play a~rather or very important role in the improvement of their language skills. Less than 5\% of respondents believe that the university should not play any role in the German language training of its students.

%\clearpage\newpage

\popis{Table 12: Perception of the university's role in the improvement of German language proficiency}
\tabcolsep 8.5pt
\footnotesize\zactab{|l|c|c|c|c|}\hline
\bf Category 		&	\mc{4}{|c|}{\bf Frequency table}	\\\cline{2-5}
				&\bf	Frequency 	&\bf	Cumulative 	&\bf	Relative 		&\bf	Cumulative relative \\
				&				&\bf	frequency 		&\bf	frequency (\%)	&\bf	frequency (\%)\\\hline
completely unimportant &12&12&4.67&4.67\\\hline
rather unimportant 	&66&78&25.68&30.35\\\hline
rather important 	&125&203&48.64&78.99\\\hline
very important 	&42&245&16.34&95.33\\\hline
n/a 	&12&257&4.67&100.00\\\hline
\kontab\\[6mm]


%%%%%%%%%%%%%%%%%%%%%%% upravit
Table 13 shows what types of subjects students are interested in in connection with the improvement of their German language skills. 163 of the respondents (i.e., 64\%) confirm they would like to improve their knowledge of general German (``definitely yes" and ``rather yes"). 152 of the respondents (nearly 50\%) are interested in an offer of specialized courses, such as ``Business German", and 104 of the respondents (40\%) would welcome it if specialized subjects (e.g., management, marketing, finance) were taught in German.%\\

\popis{Table 13. Interest in the teaching of subjects in the process of improvement of German language proficiency}
\tabcolsep 11.9pt
\footnotesize\zactab{|l|c|c|c|c|}\hline
\bf Category 		&	\mc{4}{|c|}{\bf Frequency table}	\\\cline{2-5}
				&\bf	Frequency 	&\bf	Cumulative 	&\bf	Relative 		&\bf	Cumulative relative \\
				&				&\bf	frequency 		&\bf	frequency (\%)	&\bf	frequency (\%)\\\hline
\mc{5}{|l|}{General language training}\\\hline
definitely not 	&11&11&4.28&4.28\\\hline
rather not 	&29&40&11.28&15.56\\\hline
rather yes 	&64&104&24.90&40.46\\\hline
definitely yes 	&99&203&38.52&78.98\\\hline
n/a 	&54&257&21.02&100.00\\\hline
\mc{5}{|l|}{language training -- specialized courses (e.g., business German, legal German)}\\\hline
definitely not 	&14&14&5.45&5.45\\\hline
rather not 	&29&43&11.28&16.73\\\hline
rather yes 	&103&146&40.08&56.81\\\hline
definitely yes 	&49&195&19.067&75.88\\\hline
n/a 	&62&257&24.12&100.00\\\hline
\mc{5}{|l|}{specialized subjects taught in German (e.g., management, marketing, finance)}\\\hline
definitely not 	&25&25&9.73&9.73\\\hline
rather not 	&49&74&19.07&28.80\\\hline
rather yes 	&77&151&29.96&58.76\\\hline
definitely yes 	&27&178&10.51&69.27\\\hline
n/a 	&79&257&30.73&100.00\\\hline
\kontab

\clearpage\newpage

Statistically significant differences between the perception of the role played by universities in the improvement of German language proficiency and socio-demographic characteristics were also examined. The following statistically significant differences were found to exist:



{\leftskip 5mm
      a) A~statistically significant difference was found to exist between ``{\bf the perception of the importance of the role played by the university in the improvement of German language proficiency}" and ``{\bf the respondent's interest in the improvement of German language proficiency through general language courses}" ($\chi^2=18.43$; $df=9$; $p=0.030$).\\\par}

Of the 132 respondents who perceive the role played by the university in the improvement of their German language skills as rather or very important, 124 (i.e., 94\%) also state they would like to improve their German language skills through general language courses (see Table 14). Of the options available, the highest number of respondents checked definitely yes (99 persons, i.e., 59\% of the total number of responses).\\[-2mm]

\popis{Table 14: Contingency table -- relationship between the perception of the importance of the role played by the university in the improvement of German language proficiency and the respondent's interest in the improvement of German language proficiency through general language courses}
\footnotesize\zactab{|c|c|c|c|c|c|c|}\hline
\bf Perception		&\bf	Frequency 	&	\mc{5}{|c|}{\bf Interest in Improvement of German Language Proficiency}	\\
\bf of the Role Played&				&	\mc{5}{|c|}{\bf  (General Language Courses)} 	\\\cline{3-7}
\bf by University	&				&	definitely 		&	rather		&	rather 	&	definitely 	&  sum\\
				&				&	not			&	not			&	yes		&	yes		&	\\\hline
unimportant 	&absolute (relative)&0 (0\%)&0 (0\%)&3 (2\%)&1 (1\%)&4 (3\%)\\\hline
rather unimportant 	&absolute (relative)&0 (0\%)&0 (0\%)&20 (12\%)&13 (8\%)&33 (20\%)\\\hline
rather important 	&absolute (relative)&1 (1\%)&5 (3\%)&32 (19\%)&59 (35\%)&97 (58\%)\\\hline
very important 	&absolute (relative)&0 (0\%)&2 (1\%)&7 (4\%)&26 (15\%)&35 (21\%)\\\hline
total 	&absolute (relative)&1 (1\%)&7 (4\%)&62 (37\%)&99 (59\%)&169 (100\%)\\\hline
\kontab

{\small\fntit{Note: The information on relative number is based on the total number of answers ($N=169$).}}\\%[-2mm]

{\leftskip 5mm
      b) A~statistically significant difference was found to exist between ``{\bf the perception of the importance of the role played by the university in the improvement of German language proficiency}" and ``{ \bf the respondent's interest in the improvement of German language proficiency through specialized language courses}" ($\chi^2=27.44$; $df=6$; $p<0.05$).\\\par}

Of the 136 respondents who perceive the role played by the university in the improvement of their German language skills as rather or very important, 125 (i.e., 92\%) state they would like to improve their German language skills through specialized language courses (see Table 15). Of the options available, the highest number of respondents checked rather yes (103 persons, i.e., 64\% of the total number of answers). The definitely yes option followed (48 persons, i.e., 30\% of the total number of answers).

\clearpage\newpage

\popis{Table 15: Contingency table -- relationship between the perception of the importance of the role played by the university in the improvement of German language proficiency and the respondent's interest in the improvement of German language proficiency through specialized language courses}
\footnotesize\zactab{|c|c|c|c|c|c|c|}\hline
\bf Perception		&\bf	Frequency 	&	\mc{5}{|c|}{\bf Interest in Improvement of German Language Proficiency}	\\
\bf of the Role Played&				&	\mc{5}{|c|}{\bf  (Specialized Language Courses)} 	\\\cline{3-7}
\bf by University	&				&	definitely 		&	rather		&	rather 	&	definitely 	&  sum\\
				&				&	not			&	not			&	yes		&	yes		&	\\\hline
unimportant 	&absolute (relative)&0 (0\%)&0 (0\%)&1 (1\%)&0 (0\%)&1 (1\%)\\\hline
rather unimportant &absolute (relative)&0 (0\%)&0 (0\%)&22 (14\%)&3 (2\%)&25 (15\%)\\\hline
rather important &absolute (relative)&0 (0\%)&8 (5\%)&68 (42\%)&23 (14\%)&99 (61\%)\\\hline
very important &absolute (relative)&0 (0\%)&3 (2\%)&12 (7\%)&22 (14\%)&37 (23\%)\\\hline
total 	&absolute (relative)&0 (0\%)&11 (7\%)&103 64\%)&48 (30\%)&162 (100\%)\\\hline
\kontab

{\small\fntit{Note: The information on relative number is based on the total number of answers ($N=162$).}}\\[-2mm]

%\clearpage\newpage

{\leftskip 5mm
      c) A~statistically significant difference was found to exist between ``{\bf the perception of the importance of the role played by the university in the improvement of German language proficiency}" and ``{\bf the respondent's interest in the improvement of German language proficiency through subjects taught in German}" ($\chi^2=38.22$; $df=6$; $p<0.05$).\\\par}

Of the 126 respondents who perceive the role played by the university in the improvement of their German language skills as rather or very important, 87 (i.e., 69\%) state they would like to improve their German language skills through specialized subjects taught in German (see Table 16). Of the options available, the highest number of respondents checked rather yes (76 persons, i.e., 54\% of the total number of answers). The rather not (36 persons, i.e., 35\% of the total number of answers) and definitely yes (26 persons, i.e., 18\% of the total number of answers) options followed.
      
\popis{Table 16: Contingency table -- relationship between the perception of the importance of the role played by the university in the improvement of German language proficiency and the respondent's interest in the improvement of German language proficiency through subjects taught in German}
\tabcolsep 5.5pt
\footnotesize\zactab{|c|c|c|c|c|c|c|}\hline
\bf Perception		&\bf	Frequency 	&	\mc{5}{|c|}{\bf Interest in improvement of German language proficiency}	\\
\bf of the Role Played&				&	\mc{5}{|c|}{\bf  (specialized subjects)} 	\\\cline{3-7}
\bf by University	&				&	definitely 		&	rather		&	rather 	&	definitely 	&  sum\\
				&				&	not			&	not			&	yes		&	yes		&	\\\hline
unimportant 	&absolute (relative)&0 (0\%)&0 (0\%)&0 (0\%)&0 (0\%)&0 (0\%)\\\hline
rather unimportant &absolute (relative)&0 (0\%)&0 (0\%)&13 (9\%)&2 (1\%)&15 (11\%)\\\hline
rather important &absolute (relative)&3 (2\%)&34 (24\%)&45 (32\%)&9 (6\%)&91 (65\%)\\\hline
very important &absolute (relative)&0 (0\%)&2 (1\%)&18 (13\%)&15 (11\%)&35 (25\%)\\\hline
total &absolute (relative)&3 (2\%)&36 (25\%)&76 (54\%)&26 (18\%)&162 (100\%)\\\hline
\kontab

{\small\fntit{Note: The information on relative number is based on the total number of answers ($N=162$).}}\\[-2mm]

{\leftskip 5mm
      d) A~statistically significant difference was found to exist between the ability to communicate in German and the perception of the role played by a~business and management university in the improvement of German language proficiency ($\chi^2=44.76$; $df=18$; $p<0.05$).\par}

%\clearpage\newpage

Respondents who attained levels A2 or higher in German in their self-evaluation under the Common European Framework of Reference for Language perceive the role played by the university in the improvement of German language proficiency as rather important (63\%) or very important (17\%). The role played by the university in this regard is considered rather or completely unimportant by 18\% of the respondents (see Table 17).

\popis{Table 17: Contingency table -- relationship between the ability to communicate in German and the perception of the role played by a~business and management university in the improvement of German language proficiency}
\tabcolsep 11.8pt
\footnotesize\zactab{|l|c|c|}\hline
\bf Perception of the Role Played 	&\bf	Frequency 	&\bf	Ability to Communicate in German \\
\bf by University				&				&\bf	(attained language skill levels A2--C2)\\\hline
unimportant 	&absolute (relative)&2 (2\%)\\\hline
rather unimportant &absolute (relative)&19 (16\%)\\\hline
rather important &absolute (relative)&76 (63\%)\\\hline
very important &absolute (relative)&21 (17\%)\\\hline
total &absolute (relative)&121 (100\%)\\\hline
\kontab

{\small\fntit{Note: The information on relative number is based on the total number of answers ($N=121$).}}\\[-2mm]

{\leftskip 5mm
      e) A~statistically significant difference was found to exist between the perception of the importance of German for success and professional growth on the current labour market in the Czech Republic, and interest in the improvement of German language proficiency through general language courses ($\chi^2=20.449$; $df=9$; $p=0.0153$).\\[-2mm]\par}

As can be seen in Table 18, of the 136 respondents who consider German rather or very important, 122 (i.e., nearly 90\%) stated they would like to improve their German language skills through general language courses (definitely yes or rather yes options). Respondents who refer to German as rather unimportant for professional life (52 persons) are also clearly interested in improving their general language skills (38 respondents, i.e., 73\%, opted for rather yes or definitely yes).
      
\popis{Table 18: Contingency table -- relationship between the perception of the importance of German for success and professional growth on the current labour market in the Czech Republic, and respondent's interest in the improvement of German language proficiency through general language courses}
\tabcolsep 4.5pt
\footnotesize\zactab{|c|c|c|c|c|c|c|}\hline
\bf Perception		&\bf	Frequency 	&	\mc{5}{|c|}{\bf Interest in improvement of German language}	\\
\bf of the importance&				&	\mc{5}{|c|}{\bf proficiency (general language courses)} 	\\\cline{3-7}
\bf of German		&				&	definitely 		&	rather		&	rather 	&	definitely 	&  sum\\
				&				&	not			&	not			&	yes		&	yes		&	\\\hline
completely unimportant&absolute (relative)&0 (0\%)&2 (1\%)&1 (1\%)&0 (0\%)&3 (2\%)\\\hline
rather unimportant&absolute (relative)&0 (0\%)&14 (7\%)&20 (10\%)&18 (9\%)&52 (27\%)\\\hline
rather important&absolute (relative)&1 (1\%)&12 (6\%)&35 (18\%)&65 (34\%)&113 (59\%)\\\hline
very important&absolute (relative)&0 (0\%)&1 (1\%)&7 (4\%)&15 (8\%)&23 (12\%)\\\hline
total&absolute (relative)&1 (1\%)&29 (15\%)&63 (33\%)&98 (51\%)&191 (100\%)\\\hline
\kontab

{\small\fntit{Note: The information on relative number is based on the total number of answers ($N=191$).}}\\[-2mm]

{\leftskip 5mm
      f) A~statistically significant difference was found to exist between the perception of the importance of German for success and professional growth on the current labour market in the Czech Republic, and interest in the improvement of German language proficiency through specialized language courses ($\chi^2=19.561$; $df=9$; $p=0.021$).\par}

As can be seen in Table 19, of the 133 respondents who consider German rather or very important, 114 (i.e., 86\%) stated they would like to improve their German language skills through general language courses (rather or definitely yes options). Respondents who refer to German as rather unimportant for professional life (45 persons) mostly also express an interest in improving their general language skills through specialized courses (35\,respondents opted for rather yes or definitely yes).
      
\popis{Table 19: Contingency table -- relationship between the perception of the importance of German for success and professional growth on the current labour market in the Czech Republic, and respondent's interest in the improvement of German language proficiency through specialized language courses}
\tabcolsep 4.1pt
\footnotesize\zactab{|c|c|c|c|c|c|c|}\hline
\bf Perception		&\bf	Frequency 	&	\mc{5}{|c|}{\bf Interest in improvement of German language}	\\
\bf of the importance&				&	\mc{5}{|c|}{\bf proficiency (specialized language courses)} 	\\\cline{3-7}
\bf of German		&				&	definitely 		&	rather		&	rather 	&	definitely 	&  sum\\
				&				&	not			&	not			&	yes		&	yes		&	\\\hline
completely unimportant&absolute (relative)&0 (0\%)&2 (1\%)&1 (1\%)&0 (0\%)&3 (2\%)\\\hline
rather unimportant&absolute (relative)&0 (0\%)&10 (6\%)&28 (15\%)&7 (4\%)&45 (25\%)\\\hline
rather important&absolute (relative)&2 (1\%)&13 (7\%)&66 (36\%)&29 (16\%)&110 (61\%)\\\hline
very important&absolute (relative)&0 (0\%)&4 (2\%)&7 (4\%)&12 (7\%)&23 (13\%)\\\hline
total&absolute (relative)&2 (1\%)&29 (16\%)&102 (56\%)&48 (27\%)&181 (100\%)\\\hline
\kontab

{\small\fntit{Note: The information on relative number is based on the total number of answers ($N=181$).}}\\[-2mm]

{\leftskip 5mm
      g) A~statistically significant difference was found to exist between the perception of the importance of German for success and professional growth on the current labour market in the Czech Republic, and interest in the improvement of German language proficiency through specialized subjects taught in German ($\chi^2=25.622$; $df=9$; $p=0.016$).\\[-2mm]\par}

As can be seen in Table 20, of the 122 respondents who consider German rather or very important, 81 (i.e., 66\%) expressed an interest in improving their German language proficiency through subjects taught in German (rather or definitely yes options). Respondents who refer to German as rather unimportant for professional life (35 persons) mostly also express an interest in improving their general language skills through specialized subjects taught in German (22 respondents, or 63\%, opted for rather yes or definitely yes).
      
\popis{Table 20: Contingency table -- relationship between the perception of the importance of German for success and professional growth on the current labour market in the Czech Republic, and respondent's interest in the improvement of German language proficiency through subjects taught in German}
\tabcolsep 4.4pt
\footnotesize\zactab{|c|c|c|c|c|c|c|}\hline
\bf Perception		&\bf	Frequency 	&	\mc{5}{|c|}{\bf Interest in improvement of German language}	\\
\bf of the importance&				&	\mc{5}{|c|}{\bf proficiency (offer of specialized subjects)} 	\\\cline{3-7}
\bf of German		&				&	definitely 		&	rather		&	rather 	&	definitely 	&  sum\\
				&				&	not			&	not			&	yes		&	yes		&	\\\hline
completely unimportant&absolute (relative)&0 (0\%)&3 (2\%)&0 (0\%)&0 (0\%)&3 (2\%)\\\hline
rather unimportant&absolute (relative)&0 (0\%)&13 (8\%)&20 (13\%)&2 (1\%)&35 (22\%)\\\hline
rather important&absolute (relative)&7 (4\%)&30 (19\%)&48 (30\%)&15 (9\%)&100 (63\%)\\\hline
very important&absolute (relative)&1 (1\%)&3 (2\%)&8 (5\%)&10 (6\%)&22 (14\%)\\\hline
total&absolute (relative)&8 (5\%)&49 (31\%)&76 (48\%)&27 (17\%)&160 (100\%)\\\hline
\kontab

{\small\fntit{Note: The information on relative number is based on the total number of answers ($N=160$).}}

\clearpage\newpage

\podnadpis{Conclusion}
\vspace*{-4mm}
The results of the survey show that the ability of students of Czech business and management universities to communicate in German is not very high. A~mere 7.41 of the respondents is able to communicate orally and in writing at B2--C2 levels under the Common European Framework of Reference for Language. These values are comparable to the results of the isea survey (2010) according to which less than 5\% of the Czech population can communicate in German ``very well". In this context, it is appropriate to draw attention to the development indicated by statistical data of MEYS (2012): the share of students studying two and more foreign languages has been declining continually in 2007--2011, from 25\% (2007) to 17\% (2011), and the status of German as the second foreign language most frequently studied in the Czech Republic weakened slightly. Instead of the original 25\% in 2007, less than 20\% of all students studied this language in 2011. Special Eurobarometer 386 (2012) actually points out the declining portion of Czech population conversant in at least one foreign language as compared to 2005. The document further stresses, \fntit{inter alia}, the significant decline in the number of respondents in the Czech Republic able to communicate in German.

It can be noted that the declining portion of Czech population, especially students at business and management universities, conversant in two or more foreign languages, definitely represents an unfavourable trend which is moreover in direct conflict with the policy of the European Commission which in its document, An Action Plan 2004--2006, calls on EU citizens to learn foreign languages, and recommends they master two further foreign languages, specifically, English and the language of their neighbouring country (European Commission 2003).

If we focus on factors affecting the level of German language proficiency in the case of Czech university students of business and management, it is apparent that the school environment has a~decisive influence on the respondents' ability to communicate in German (the opportunity to learn the language, compulsory subject, motivation to study the language). Factors such as contemplations of future career, parents' wishes, relationship to the country and culture, or study or work stay, play a~rather marginal role in the acquisition of German language proficiency. The consistency of survey results of isea (2010) can be confirmed in this regard as well: it demonstrates the ``fundamental" importance of school in the learning of foreign languages. The extraordinary importance of language education is also stressed by the Report on the implementation of the Action Plan ``Promoting language learning and linguistic diversity" addressing the implementation of An Action Plan 2004--2006 (2009), which commends selected EU member states for their efforts to innovate their national educational systems, for instance, by introducing foreign language training of preschool children, implementation of the Content and Language Integrated Learning/CLIL model, expansion of foreign language training at the secondary level of schooling, or investments into teacher training. Lasagabaster (2008) also stresses challenges faced by European educational systems in light of the growing importance of foreign language training, since ``there is a~dire need to educate multilingual and multicultural citizens in a~context where the linguistic consequences of globalization are more and more evident" (Lasagabaster 2008). Given the Czech Republic's historical, cultural and economic ties to German-speaking countries, we are of the opinion that German should logically be the second language after English, the teaching of which would be pro-actively supported by the Czech system of education at all its levels.

The survey results further indicate that nearly three-quarters of the respondents view German as a~language rather important or very important for their success on the current Czech labour market, even though nearly one-half of the respondents voiced the opinion that knowledge of English was sufficient. This result confirms that the perception of German as a~language useful for professional purposes and professional growth is incontestable in Czech conditions. While the perception of the importance of German declines with age, the view that German is rather or very important for professional purposes prevails across age groups. Another important finding is that, regardless of the respondents' work experience or lack thereof, and the type of study they are enrolled in, the prevailing opinion is that German is rather or very important for success on the labour market in the Czech Republic. Another survey result according to which full-time students mostly do not subscribe to the opinion that knowledge of English is sufficient for career success is rather contradictory; however, more than one-half of students enrolled in a~combined type of study holds this view. Students enrolled in a~combined type of study in particular can presumably be expected to have greater work experience and thus better knowledge of the requirements encountered in corporate practice. However, despite this contradiction, it can be noted that the survey results emphasize the conviction of a~large segment of the respondents that language skills in both of those world languages need to be improved. German thus is, in addition to English, a~language the knowledge of which, according to a~large segment of the respondents, increases the likelihood of success on the Czech labour market. The isea survey (2010) yielded similar results. It addressed, \fntit{inter alia}, the motivation for foreign language learning, and arrived at the conclusion that the Czech population is mainly motivated to learn German by practical concerns (``it is in particular a~language important for one's profession").

The respondents are thus fully aware of the importance of knowledge of two or more foreign languages, especially English and German, as a~prerequisite to success on the labour market, and a~number of other studies conducted at European level considers foreign language skills to be one of the key factors of the success of individuals on the labour market and companies in global competition (e.g., Baur 2009, Gudauner 2009, Lasagabaster 2008, Loi 2009, P�rnbacher 2009). These studies point out risks inherent in the belief that the mastery of English for communication on the European labour market is sufficient, and stress that employers often require foreign language skills from their employees. Foreign language competence is deemed to be the third most important competence after informatics and manufacturing, and marketing and sales. The processes of internationalization and globalization clearly give preference to the employment of persons conversant in several European languages, and empirical data testifies to the extraordinarily important position of German as the second foreign language in corporate practice, especially in regions that are geographically close to German speaking countries (Benelux, Poland, Croatia, Slovenia, Czech Republic, Slovakia and Hungary). Companies are well aware that German communication skills also open markets above and beyond German-speaking countries. In this context, the National Centre for Languages (2006) points out damage in the form of loss of prospective business and unsuccessful entry on foreign markets, caused by a~lack of adequate language skills on the part of the employees of European SMEs.

The above findings may thus be interpreted as terms of reference for politics and national systems of education at the level of foreign language teaching. In the Czech context, it specifically means investing in the teaching of at least two foreign languages, the preferred combination being English and German, at all levels of schooling, given the above-mentioned factors, such as geographic location, common historical and cultural roots, and last but not least, intense economic relations. University students, i.e., future employees, should consider the fact that, apart from other things, the labour market in German-speaking countries offers substantial opportunities for career building and professional growth, when thinking of what languages to study.

Other survey results convincingly testify to the fact that German is definitely not on the margin of interest for Czech university students of business and management. Nearly two-thirds of the respondents believe that business and management universities play a~rather important or very important role in the improvement of German language proficiency. As regards the improvement of their German language skills, students are interested in general German courses (64\%), specialized courses, such as ``Business German" (nearly 50\%), and 40\% of the respondents would welcome specialized subjects (e.g., management, marketing, finance) taught in German. It is thus apparent that a~great majority of the students is aware of the importance of knowledge of the German language for their professional growth. This result should certainly be taken into account when the curricula for the respective study programmes and courses are drawn up. German at universities should not gain the status of a~subject of marginal interest to students. Universities ought to respond to this challenge, and it appears appropriate to point out the opportunities inherent in the implementation of the Content and Language Integrated Learning (CLIL) concept. For instance, Lasagabaster (2008), who examined the implementation of this concept in the Basque Country, concluded that the ``CLIL approach is successful and helps to improve students' foreign language competence even in bilingual contexts where English has little social presence". Similarly, a~review of studies conducted by Dalton Puffer (2007) in German conditions stresses ``CLIL advantages concerning receptive skills, vocabulary and fluency". The teaching of specialized subjects in German may thus be one of the ways of strengthening the German language proficiency of university students of business and management in the Czech Republic, and increase their chances of success on the Czech and European labour markets. 

In conclusion, the authors would like to voice their conviction that the results of this study may be viewed as an important contribution to increasing the current level of knowledge of those factors that influence German language skills of Czech university students of business and management, and by extension their potential success on the labour markets at home and abroad. The study further proposes potential directions which language training could follow in order to improve the German language proficiency of business and management university graduates.%\\

\clearpage\newpage

\cast{References}
Amon, U. (1991). \fntit{Die internationale Stellung der Deutschen Sprache}. Berlin -- New York: Walter de Gruyter.

Baur, S. (2009). Sprachenlernen und Kompetenzen in einem mehrsprachigen europ�ischen Gebiet. In Autonome Provinz Bozen-S�dtirol (Eds.), \fntit{Sprachkompetenzen am S�dtiroler Arbeitsmarkt} (p. 59--68), Bozen.

Council of Europe. (2001). The Common European Framework of Reference for Languages, Brussels.

Council of Europe and European Parliament. (2006). \fntit{Recommendation of the European Parliament and of the Council on key competences for lifelong learning}. Retrieved on 17th March, 2013, from http://eur-lex.europa.eu/LexUriServ/LexUriServ.do?uri=CE\-LEX:32006H0962:EN:HTML.

Czech Ministry of Education, Youth and Sports/MEYS. (2012). \fntit{V� -- v�uka ciz�ho jazyka -- studenti (po�et studi�)}. Retrieved on 20th January, 2013, from http://msmt.cz.

CzechTrade. (2012). \fntit{Zahrani�n� obchod �esk� republiky v roce 2011}.  Retrieved on 20th October, 2012, from http://czechtrade.cz.

Dalton-Puffer, C. (2007). Outcomes and processes in content and language integrated learning (CLIL): current research from Europe. In Delanoy W, Volkmann L (Eds), \fntit{Future perspectives for English language teaching}. Heidelberg: Carl Winter.

European Commission. (2012). \fntit{Special Eurobarometer 386 -- Europeans and their languages}. Retrieved on 12th May, 2013, from http://ec.europa.eu/public\_opinion/archi\-ves/ebs/e\-bs\_386\_en.pdf.

European Commission. (2006). \fntit{Special Eurobarometer 243 -- Europeans and their languages}. Retrieved on 12th May, 2013, from http://ec.europa.eu/public\_opinion/archi\-ves/ebs/e\-bs\_243\_en.pdf.

European Commission. (2010). \fntit{Work programme on the follow-up of the objectives of education and training systems in Europe}. Retrieved on 20th May, 2013, from http://europa.eu/\\legislation\_summaries/education\_training\_youth/general\_framework/c11086\_en.htm.

European Commission. (2009). \fntit{Report on the implementation of the Action Plan ``Promoting language learning and linguistic diversity''}. Retrieved on 11th March, 2013, from http://europa.eu/legislation\_summaries/education\_training\_youth/lifelong\_learning/c1106\-8\_en.htm.

European Commission. (2003). \fntit{Promoting Language Learning and Linguistic Diversity: An Action Plan 2004--2006}. Retrieved on 30th December, 2012, from http://eur-lex.europa.eu/LexUriServ/LexUriServ.do?uri=COM:2003:0449:FIN:en:PDF.

European Commission. (1995). \fntit{White Paper on Education and Training: Teaching and Learning, Towards the Learning Society}. Retrieved on 1st December, 2012, from http://aei.\\pitt.edu/1132/1/education\_train\_wp\_COM\_95\_590.pdf.

Greiner, U. (2010). Ist Deutsch noch zu retten? \fntit{Die Zeit}, 2010 (27), 1--8.

Gudauner, K. (2009). Rahmenbedingungen f�r den Sprachenerwerb in der Produktionswelt. In Autonome Provinz Bozen-S�dtirol (Eds.), \fntit{Sprachkompetenzen am S�dtiroler Arbeitsmarkt} (p. 9--13), Bozen.

\clearpage\newpage

H�ffe, O. (1999). Ein Gesellschaftsvertrag f�r Europa? Ein Versuch in dreizehn Thesen. In Anselm, E./Freytag, A./Marschitz, W./Marte, B. (Eds.), \fntit{Die neue Ordnung des Politischen. Die Herausforderungen der Demokratie am Beginn des 21. Jahrhunderts} (p. 267--278), Frankfurt/ M./New York: Campus.

Institute for Social and Economic Analyses/isea. (2010). \fntit{Cizojazy�n� kompetence �esk� populace: n�m�ina ve srovn�n� s~jin�mi jazyky}. Retrieved on 1st March, 2011, from http://www.isea-cz.org/ExpertsFile/tabid/89/Default.aspx.

Lasagabaster, D. (2008). Foreign language competence in content and language integrated courses. \fntit{The Open Applied Linguistics Journal}, 2008 (1), 31--42.

Loi, M. (2009). Sprachkompetenzen in S�dtirol: Deutung der Ergebnisse unter dem Blick\-winkel der Humankapitaltheorie. In Autonome Provinz Bozen-S�dtirol (Eds.), \fntit{Sprachkompetenzen am S�dtiroler Arbeitsmarkt} (p. 49--57), Bozen.

M�ller, H., M. (2003). \fntit{Schlaglichter der deutschen Geschichte}. Bonn: Bundeszentrale f�r politische Bildung.

National Centre for Languages. (2006). \fntit{ELAN: Effects on the European Economy of Shortages of Foreign Language Skills in Enterprise}. Retrieved on 1st March, 2011, from https://www.llas.ac.uk/news/2772.

P�rnbacher, K. (2009). Sprachenlernen am Arbeitsplatz: Zwang, Chance oder Recht? In Autonome Provinz Bozen-S�dtirol (Eds.), \fntit{Sprachkompetenzen am S�dtiroler Arbeitsmarkt} (p. 23--33), Bozen.

P�rnbacher, K. (2009). Sprachkompetenzen als Wettbewerbsfaktor f�r S�dtirols Betriebe. In Autonome Provinz Bozen-S�dtirol (Eds.), \fntit{Sprachkompetenzen am S�dtiroler Arbeitsmarkt} (p. 35--42), Bozen.

Riemeck, R. (1997). \fntit{Mitteleuropa: Bilanz eines Jahrhunderts}. Stuttgart: Verlag Engel \&~Co.

Ritzer, G. (2011). Die McDonaldisierung von Gesellschaft und Kultur. In Stephan Moebius \& Dirk Quadflieg (Eds.), \fntit{Kultur. Theorien der Gegenwart} (p. 371--378). Wiesbaden: VS Verlag f�r Sozialwissenschaften.

